\documentclass[a4paper,UKenglish]{lipics-v2016}
\usepackage{microtype}
\usepackage{amssymb,amsmath}
\usepackage{graphics,graphicx,color}
\usepackage{enumerate,paralist,hyperref}
\usepackage{caption,subcaption,xspace}
\usepackage{todonotes} 
\usepackage{times,setspace} 

%%%% Title of contribution
% ==================================================================
\title{Characterizations of Optimal 2- and 3-planar Graphs}
\titlerunning{Characterizations of Optimal 2- and 3-planar Graphs}
\author[1]{Michael~A.~Bekos}
\author[1]{Michael~Kaufmann}
\author[2]{Chrysanthi~N.~Raftopoulou}
% ==================================================================
\affil[1]{
Wilhelm-Schickhard-Institut f\"ur Informatik, Universit\"at T\"ubingen, Germany\\
\texttt{\{bekos,mk\}@informatik.uni-tuebingen.de}}
\affil[2]{
School of Applied Mathematical \& Physical Sciences, NTUA, Greece\\
\texttt{crisraft@mail.ntua.gr}}
% ==================================================================

% ==================================================================
\authorrunning{Michael~A.~Bekos, Michael~Kaufmann, Chrysanthi~N.~Raftopoulou}
% ==================================================================

\Copyright{Michael~A.~Bekos, Michael~Kaufmann, Chrysanthi~N.~Raftopoulou}
\subjclass{G.2.2 Graph Theory}% mandatory: Please choose ACM 1998 classifications from http://www.acm.org/about/class/ccs98-html . E.g., cite as "F.1.1 Models of Computation".
\keywords{$k$-planar graphs, characterization, graph drawing}%

%%%%%%%%%%%%%%%%%%%%%%%%%%%%%%%%%%%%%%%%%%%%%%%%%%%%%%%%%
%Editor-only macros (do not touch as author)
\EventEditors{Boris Aronov, Matthew Katz}
\EventNoEds{1}
\EventLongTitle{33rd International Symposium on Computational Geometry}
\EventShortTitle{SoCG 2017}
\EventAcronym{SoCG}
\EventYear{2017}
\EventDate{July 4-7, 2017}
\EventLocation{Brisbane, Australia}
\EventLogo{}
\SeriesVolume{}
\ArticleNo{X} % Your paper number goes here! (not your EasyChair submission number)
%%%%%%%%%%%%%%%%%%%%%%%%%%%%%%%%%%%%%%%%%%%%%%%%%%%%%%%%%

\newtheorem{property}{Property}

\newcommand{\pe}{potential edge\xspace}
\newcommand{\pes}{potential edges\xspace}
\newcommand{\pp}{potential empty cycle\xspace}
\newcommand{\pr}{potential region\xspace}
\newcommand{\pb}{potential boundary\xspace}

\newcommand{\parallelsum}{\mathbin{/\mkern-4mu/}}
\newcommand{\ppp}[2]{\angle [#1,#2]}
\newcommand{\ppc}[2]{\parallelsum [#1,#2]}

\begin{document}
\maketitle

% ============================================================================
\begin{abstract}
$k$-planarity is a concept that extends the class of planar graphs by allowing graphs that can be drawn with at most $k$ crossings on each edge. While for $1$-planar graphs, the \emph{optimal} graphs, i.e., those with the maximum number of edges, have been completely characterized, this was not the case for $k \geq 2$. For such cases, in particular for $k=2,3$ and $4$, upper bounds on the number of edges have been developed for the case of simple graphs by Pach and Toth, Pach et al.\ and Ackerman. Since then the bounds are known to be achieved by graphs with a very regular structure; the set of uncrossed edges form pentagonal or hexagonal faces respectively, which then are completely by $5$ more edges in the pentagons for the $2$-planar case, and by $8$ more edges in the hexagons for the $3$-planar case. Unfortunately, non-homotopic parallel edges have to be allowed. Previously, we have shown that the known bounds for the simple graphs hold also for the case of non-simple graphs with non-homotopic parallel edges and loops. In this paper, we prove that only those graphs achieve the bound of $5n-10$ for $2$-planar and the $5.5n -11$ bound for $3$-planar graph and provide a complete characterization of optimal non-simple $2$- and $3$-planar graphs.
\end{abstract}
% ============================================================================

% ============================================================================
\section{Introduction}
\label{sec:introduction}
% ============================================================================

The gradually improvement of the well-known lemma of crossing numbers \cite{ACNS82,Lei83}, but also recent interest in graphs beyond planarity gave rise to multiple extensions of planar graphs allowing edge crossing in some restricted local configurations. The most prominent is the concept of $k$-planarity; here each edge can be crossed at mot $k$ times. Earlier work mainly considered the case $k=1$, i.e., $1$-planar graphs. We mention here the coloring problem of 1-planar graphs by Ringel~\cite{Ringel65} and the first density bounds by Bodendiek, Schumacher and Wagner~\cite{BSW84}. Later on, Suzuki \cite{DBLP:journals/siamdm/Suzuki10} gave simple rules how to generate maximal $1$-planar graphs. Other works considered maximal 1-planar graphs~\cite {DBLP:conf/gd/BrandenburgEGGHR12} and recognition of $1$-planar graphs~\cite{DBLP:journals/corr/Brandenburg16a}. But also subclasses have been considered like \emph{IC-planar graphs}, where for each vertex, only one incident edge can be crossed \cite{DBLP:conf/gd/BrandenburgDEKL15}, and several others on outer-1-planarity where all the vertices have to be adjacent to the outer face as being required in the corresponding model of outer-planarity~\cite{DBLP:journals/algorithmica/HongEKLSS15,DBLP:journals/jgaa/GiacomoLM15,DBLP:journals/algorithmica/AuerBBGHNR16}.

For larger $k \geq 2$, the paper of Pach and Toth~\cite{PachT97} provided significant progress, as it gives techniques for upper bounds on the edge density for simple $k$-planar graphs, which then lead to upper bounds of $5n -10$ for $2$-planar, $6n -12$ for $3$-planar and $7n-13$ for $4$-planar graphs. For larger $k$, the authors provide a bound of $4.1 \sqrt k n$. While the bound for $2$-planar graphs is tight, the other bounds have been improved to $5.5n - 11$ \cite{PachRTT06} and $6n-12$ \cite{DBLP:journals/corr/Ackerman15} for simple $3$- and $4$-planar graphs, respectively. In our companion paper~\cite{BKR16}, we generalize the result and the bound of Pach et al.~\cite{PachRTT06} to non-simple graphs, where non-homotopic parallel edges as well as non-homotopic selfloops are allowed.

In this paper, we now completely characterize optimal non-simple $2$-planar and $3$-planar graphs, i.e., those that achieve the bound of $5n-10$ and $5.5n-11$ on the number of edges, respectively. In particular, we prove that the commonly known $2$-planar graphs that achieve the upper bound of $5n-10$ are the only optimal $2$-planar graphs. Such graphs consist of a crossing-free subgraph where all faces have size $5$, which then are enhanced by $5$ more edges forming a $K_5$'s together with the boundary edges of the face (Section~\ref{sec:2planar}). For $3$-planar graphs, we can show correspondingly that optimal $3$-planar graphs have a similar simple structure (Section~\ref{sec:3planar}).

Related to this work are also other extensions of planarity, like fan-planarity \cite{KU14,DBLP:conf/gd/BekosCGHK14,DBLP:journals/tcs/BinucciGDMPST15,DBLP:conf/gd/BinucciCDGKKMT15}, quasi-planarity \cite{AAPPS97,DBLP:journals/cj/GiacomoDLMW15}, k-quasi-planarity~\cite{DBLP:conf/gd/Suk11,DBLP:journals/comgeo/SukW15}, RAC drawability~\cite{DBLP:journals/tcs/DidimoEL11,DBLP:journals/jgaa/ArgyriouBS12}, and fan-crossing-free planarity~\cite{DBLP:journals/algorithmica/CheongHKK15} 
as well as alternative geometric representations like bar-1 visibility~\cite{DBLP:conf/icaa/SultanaRRT14,DBLP:journals/jgaa/Evans0LMW14,DBLP:conf/walcom/BrandenburgEN16} and more in general bar-$k$ visibility~\cite{DBLP:journals/dm/GenesonKT14,DBLP:journals/corr/SawhneyW16}. Note that all aforemention classes of nearly-planar graphs have also linear number of edges (except for the classes of $k$-quasi planar and bar-$k$ visibile graphs, whose number of edges also depends on the parameter $k$). 


% ============================================================================
\section{Preliminaries}
\label{sec:preliminaries}
% ============================================================================

Let $G$ be a (not necessarily simple) \emph{topological graph}, that is, $G$ is a graph drawn on the plane, so that the vertices of $G$ are distinct points in the plane, its edges are Jordan courves joining the corresponding pairs of points, and:%
%
\begin{inparaenum}[(i)]
\item no edge passes through a vertex different from its endpoints, 
\item no edge crosses itself and 
\item no two edges meet tangentially.
\end{inparaenum}
Let $\Gamma(G)$ be such a drawing of $G$.

An edge $e$ in $\Gamma(G)$ is called a \emph{topological edge} (or simply edge, if this is clear in the context). Edge $e$ is called \emph{true-planar}, if it is not crossed by any other edge in $\Gamma(G)$. The set of all true-planar edges of $\Gamma(G)$ forms the so-called \emph{true-planar skeleton} of $\Gamma(G)$, which we denote by $\Pi(G)$. Since $G$ is not necessarily simple, we will further assume that $\Gamma(G)$ contains neither \emph{homotopic parallel edges} nor \emph{homotopic self-loops}, that is, both the bounded and the unbounded regions defined by any self-loop or by any pair of parallel edges contain at least one vertex in their interiors. For a possitive integer $s$, a cycle of length $s$ is called \emph{true-planar $s$-cycle} if it consists of true-planar edges of $\Gamma(G)$. Let $\mathcal{F}_s=\{v_1,v_2,\ldots,v_s\}$ be a facial $s$-cycle of $\Pi(G)$ with length $s \geq 3$.  The order of the vertices (and subsequently the order of the edges) of $\mathcal{F}_s$ is determined by a walk around the boundary of $\mathcal{F}_s$ in clockwise direction. Since $\mathcal{F}_s$ is not necessarily simple, a vertex (or an edge, respectively) may appear more than once in this order; see Figure~\ref{fig:non_simple_face}.

\begin{figure}[tb]
    \centering	
	\begin{minipage}[b]{.18\textwidth}
        \centering
        \includegraphics[width=\textwidth,page=1]{images/preliminaries}
        \subcaption{~}\label{fig:non_simple_face}
    \end{minipage}
	\begin{minipage}[b]{.18\textwidth}
        \centering
        \includegraphics[width=\textwidth,page=1]{images/crossing_conf}
        \subcaption{~}\label{fig:crossing_twice}
    \end{minipage}	
    \begin{minipage}[b]{.18\textwidth}
        \centering
        \includegraphics[width=\textwidth,page=2]{images/crossing_conf}
        \subcaption{~}\label{fig:crossing_twice_2}
    \end{minipage}
    \begin{minipage}[b]{.18\textwidth}
        \centering
        \includegraphics[width=\textwidth,page=3]{images/crossing_conf}
        \subcaption{~}\label{fig:crossing_adjacent}
    \end{minipage}	
    \begin{minipage}[b]{.18\textwidth}
        \centering
        \includegraphics[width=\textwidth,page=4]{images/crossing_conf}
        \subcaption{~}\label{fig:crossing_adjacent_2}
    \end{minipage}	
    \caption{%
    (a)~A non-simple face $\{v_1,\ldots,v_7\}$, where $v_6$ is identified with $v_4$.
    Different configurations used in 
    (b-c)~Lemma~\ref{lem:crossing_twice}, and 
    (d-e)~Lemma~\ref{lem:crossing_adjacent}.}
    \label{fig:2_planar_polygon_conf}
\end{figure} 

Drawing $\Gamma(G)$ is called \emph{$k$-planar} if every edge in $\Gamma(G)$ is crossed at most $k$ times. Accordingly, a graph is called \emph{$k$-planar} if it admits a $k$-planar drawing. An \emph{optimal $k$-planar} graph is a $k$-planar graph with the maximum number of edges (e.g., an optimal $1$-planar graph on $n$ vertices is a $1$-planar graph with exactly $4n-8$ edges). For an optinal $k$-planar graph $G$ on $n$ vertices, a $k$-planar drawing $\Gamma(G)$ of $G$ is called \emph{planar-maximal crossing-minimal} or simply PMCM-drawing if and only if $\Gamma(G)$ has the maximum number of true-planar edges among all $k$-planar drawings of $G$ and, subject to this restriction, $\Gamma(G)$ has also the minimum number of crossings.

\begin{lemma}
Let $\Gamma(G)$ be a PMCM-drawing of an optimal $k$-planar graph $G$ in which two edges $(u,v)$ and $(u',v')$ cross more that once. Let $c$ and $c'$ be two consecutive crossing points of them. Then, the closed region $R_{c,c'}$ that is defined by $(u,v)$, $(u',v')$ and the crossing points $c$ and $c'$ has at least one vertex in its interior.
\label{lem:crossing_twice}
\end{lemma}
\begin{proof}
For a proof by contradiction, assume that there is no vertex in $R_{c,c'}$. Denote by $nc(u,v)$ and $nc(u',v')$ the number of crossings along $(u,v)$ and $(u',v')$ that are between $c$ and $c'$, respectively (red drawn in Figure~\ref{fig:crossing_twice}). First assume that $nc(u,v) = nc(u',v')$. We will cope with the case where $nc(u,v) \neq nc(u',v')$ shortly. We proceed by redrawing edge $(u,v)$ and $(u',v')$ so to eliminate both crossings $c$ and $c'$ without affecting the $k$-planarity of $G$; see the dotted-drawn edges of Figure~\ref{fig:crossing_twice}. Of course, this contradicts the crossing minimality of $\Gamma(G)$. To complete the proof of this lemma, it remains to consider the case where $nc(u,v) \neq nc(u',v')$. Assume w.l.o.g.~that $nc(u,v) > nc(u',v')$. In this case, there is at least one other edge, say $(u'',v'')$, that crosses $(u,v)$ between $c$ and $c'$ at least twice, say at points $d$ and $d'$; refer to Figure~\ref{fig:crossing_twice_2}. Since the ``length'' between $d$ and $d'$ is shorter than the length between $c$ and $c'$, it follows that if we apply the analysis above on $(u,v)$ and $(u'',v'')$, then there will be eventually a pair of crossing edges, say $e$ and $e'$, that will have exactly the same number of crossings, that is, $nc(e) = nc(e')$. This pair of edges contradicts the crossing minimality of $\Gamma(G)$.    
\end{proof}
 

\begin{lemma}
Let $\Gamma(G)$ be a PMCM-drawing of an optimal $k$-planar graph $G$ in which two edges $(u,v)$ and $(u,v')$ incident to a common vertex $u$ cross. Let $c$ be the first crossing point of $(u,v)$ with $(u,v')$ starting from $u$. Then, the closed region $R_{c}$ that is defined by vertex $u$, edges $(u,v)$, $(u,v')$ and $c$ has at least one vertex in its interior.
\label{lem:crossing_adjacent}
\end{lemma}
\begin{proof}
For a proof by contradiction, assume that there is no vertex in $R_{c}$. Denote by $nc(u,v)$ and $nc(u,v')$ the number of crossings along $(u,v)$ and $(u,v')$ that are between $u$ and $c$, respectively (red drawn in Figure~\ref{fig:crossing_twice_2}). First assume that $nc(u,v) = nc(u,v')$. We will cope with the case where $nc(u,v) \neq nc(u,v')$ shortly. We proceed by eliminating crossing $c$ without affecting the $k$-planarity of $G$; see the dotted-drawn edges of Figure~\ref{fig:crossing_adjacent}. Of course, this contradicts the crossing minimality of $\Gamma(G)$. To complete the proof of this property, it remains to consider the case where $nc(u,v) \neq nc(u,v')$. Assume w.l.o.g.~that $nc(u,v) > nc(u,v')$. In this case, there is either one other edge of $u$, say $(u,v'')$ that crosses $(u,v)$ between $u$ and $c$, or there exists an edge $(u'',v'')$ that crosses at least twice edge $(u,v)$. By Lemma~\ref{lem:crossing_twice}, the latter case would imply that $R_{c}$ is not an empty region; a contradiction. Hence, there exists at least one other edge, say $(u,v'')$, that crosses $(u,v)$ between $u$ and $c$, say at point $d$; refer to Figure~\ref{fig:crossing_adjacent_2}. Since the ``length'' between $u$ and $d$ is shorter than the length between $u$ and $c$, it follows that if we apply the analysis above on $(u,v)$ and $(u,v'')$, then there will be eventually a pair of crossing edges incident to $u$, say $e$ and $e'$ that have exactly the same number of crossings, that is, $nc(e) = nc(e')$. This pair of edges contradicts the crossing minimality of $\Gamma(G)$.   
\end{proof}

A Jordan curve $[u,v]$ joining vertices $u$ and $v$ of $G$ is a \emph{\pe} in drawing $\Gamma(G)$ if and only if $[u,v]$ is not a homotopic self-loop in $\Gamma(G)$, that is, either $u \neq v$ or $u=v$ and there is at least one vertex in the interior and the exterior of $[u,v]$. Note that $u$ and $v$ are not necessarily adjacent in $G$. However, since each topological edge $(u,v) \in E$ of $G$ is represented by a Jordan curve in $\Gamma(G)$, it follows that $(u,v)$ is by definition a \pe of $G$ (among other ones that can potentially exist).

We say that vertices $v_1,v_2,\dots,v_k$ define a \emph{\pp} in $\Gamma(G)$, if there exist \pes $[v_i,v_{i+1}]$, for $i=1,\dots, k-1$ and \pe $[v_1,v_k]$ of $\Gamma(G)$, which%
\begin{inparaenum}[(i)]
\item do not cross with each other and
\item define a region in $\Gamma(G)$ that has no vertices in its interior.
\end{inparaenum}
%
%We refer to the (empty of vertices) interior region of a \pp as a \emph{polygonal region} (denoted by $\mathcal{R}_k$), and to its boundary as a \emph{polygonal boundary} (denoted by $\mathcal{B}_k$).

Now, consider a pair of vertices $u$ and $v$ of $G$ that are not necessarily distinct. We say that $u$ and $v$ form a \emph{corner pair} if and only if an edge $(u,u')$ incident to $u$ crosses an edge $(v,v')$ incident to $v$ in $\Gamma(G)$; see Figure~\ref{fig:corner_pair}. Let $c$ be the crossing point of $(u,u')$ and $(v,v')$. Clearly, any Jordan curve $[u,v]$ joining vertices $u$ and $v$ defines a closed region $R_{u,v}$ with edge-segments $(u,c)$ and $(v,c)$. We call $[u,v]$ \emph{corner edge} w.r.t.~$(u,u')$ and $(v,v')$ if and only if $R_{u,v}$ has no vertices of $G$ in its interior.   

\begin{property}
In a PMCM-drawing $\Gamma(G)$ of an optimal $k$-planar graph $G$ any corner edge $[u,v]$ is a potential edge.
\label{prp:corner}
\end{property}
\begin{proof}
By the definition of potential edges, it follows that the property holds when $u \neq v$. Consider now the case where $u=v$. In this case $[u,v]$ is a self-loop; see Figure~\ref{fig:corner_pair_same}. If the property does not hold, then it follows that $[u,v]$ is a self-loop with no vertices either in its interior or in its exterior. The former case clearly contradicts Lemma~\ref{lem:crossing_adjacent}. So we may assume w.l.o.g.~that there is no vertex in the exterior of $[u,v]$; see Figure~\ref{fig:corner_pair_ext}. In this case, however, it follows that the two edges $(u,u')$ and $(v,v')$ defining $[u,v]$ must cross more than once in the exterior of $[u,v]$, which leads to a contradiction with Lemma~\ref{lem:crossing_adjacent}.
\end{proof}

 \begin{figure}[tb]
    \centering		
    \begin{minipage}[b]{.16\textwidth}
        \centering
        \includegraphics[width=\textwidth,page=2]{images/preliminaries}
        \subcaption{~}\label{fig:corner_pair}
    \end{minipage}
    \begin{minipage}[b]{.16\textwidth}
        \centering
        \includegraphics[width=\textwidth,page=3]{images/preliminaries}
        \subcaption{~}\label{fig:corner_pair_same}
    \end{minipage}
    \begin{minipage}[b]{.16\textwidth}
        \centering
        \includegraphics[width=\textwidth,page=4]{images/preliminaries}
        \subcaption{~}\label{fig:corner_pair_ext}
    \end{minipage}
    \begin{minipage}[b]{.16\textwidth}
        \centering
        \includegraphics[width=\textwidth,page=5]{images/preliminaries}
        \subcaption{~}\label{fig:parallel_pair}
    \end{minipage}
	\begin{minipage}[b]{.16\textwidth}
        \centering
        \includegraphics[width=\textwidth,page=6]{images/preliminaries}
        \subcaption{~}\label{fig:parallel_pair_same}
    \end{minipage}
	\begin{minipage}[b]{.16\textwidth}
        \centering
        \includegraphics[width=\textwidth,page=7]{images/preliminaries}
        \subcaption{~}\label{fig:parallel_pair_homotopic}
    \end{minipage}
    \caption{%    
    (a-c)~vertices $u$ and $v$ form a corner pair;
    (d-e)~vertices $u$ and $v$ form a parallel pair; 
    (f)~at least one of the two potential parallel edges exists.}
    \label{fig:crossing_confs}
\end{figure} 

We say that vertices $u$ and $v$ form a \emph{parallel pair} if and only if an edge $(u,u')$ of $u$ and an edge $(v,v')$ of $v$ both cross a third edge $(w,w')$ in $\Gamma(G)$ and additionally $(u,u')$ is not identified with $(v,v')$; see Figure~\ref{fig:parallel_pair}. Let $c$ and $c'$ be the crossing points of $(u,u')$ and $(v,v')$ with $(w,w')$, respectively. Clearly, any Jordan curve $[u,v]$ joining vertices $u$ and $v$ defines a closed region $R_{u,v}$ with edge-segments $(u,c)$, $(c,c')$ and $(v,c')$. We call $[u,v]$ \emph{parallel edge} w.r.t.~$(u,u')$ and $(v,v')$ if and only if $R_{u,v}$ has no vertices of $G$ in its interior. Symmetrically we define region $R_{u',v'}$ and parallel edge $[u',v']$ w.r.t.~$(u,u')$ and $(v,v')$.

%We denote a corner (parallel) pair formed by $u$ and $v$ by $\ppp{u}{v}$ ($\ppc{u}{v}$, resp.). We have the following properties:

\begin{property}
In a PMCM-drawing $\Gamma(G)$ of an optimal $k$-planar graph $G$ at least one of parallel edges $[u,v]$, $[u',v']$ is a potential edge.
\label{prp:parallel}
\end{property}
\begin{proof}
For a proof by contradiction, assume that neither $[u,v]$ nor $[u',v']$ is a potential edge. This implies that $u=v$, $u'=v'$ and both $[u,v]$ and $[u',v']$ are self-loops that have no vertices in their interiors or their exteriors. Figure~\ref{fig:parallel_pair_homotopic} illustrates the case where both $[u,v]$ and $[u',v']$ are self-loops with no vertices in their interiors; the other cases are symmetric. It is not hard to see that $(u,u')$ and $(v,v')$ are homotopic parallel edges; a contradiction.
\end{proof}
%
We say that $(u,u')$ and $(v,v')$ are \emph{independent} if and only if both parallel edges $[u,v]$ and $[u',v']$ are \pes.

\begin{remark}
Note that in the definition of parallel pairs we required $(u,u')$ not to be identified with $(v,v')$. If this is not the case, then edge $(w,w')$ is being crossed twice by $(u,u') = (v,v')$. To make the description of our proofs simple, in Sections~\ref{sec:2planar} and \ref{sec:3planar} we will initially require that every pair of edges in $\Gamma(G)$ crosses at most once. We will call a drawing fulfilling this requirement \emph{almost-simple}. This requirement will be formally proved in both cases (see Lemmas~\ref{lem:2_planar_cross_twice} and~\ref{lem:3_planar_cross_twice}).    
\end{remark}




\input{properties.tex}
% ============================================================================
\section{Characterization of 2-planar graphs}
\label{sec:2planar}
% ============================================================================

Let $G$ be an optimal $2$-planar graph on $n$ vertices (and therefore with $5n-10$ edges). Let also $\Gamma(G)$ be a PMCM $2$-planar drawing of $G$, i.e. $\Gamma(G)$ has the maximum number of true-planar edges among all potential $2$-planar drawing of $G$ and, subject to this restriction, $\Gamma(G)$ has also the minimum number of crossings. 
%For the sake of simplicity we also momentarily assume that in $\Gamma(G)$ there is no pair of edges that cross twice, i.e. $\Gamma(G)$ is almost-simple. This assumption  will be settled soon (see Property~\ref{prp:2_planar_cross_twice}). 
In the following lemmas, we examine structural properties of a PMCM-drawing $\Gamma(G)$.

Note that in a PMCM-drawing $\Gamma(G)$, whenever edge $(w,w')$ has at least $2$ 
crossings, say $c$ and $c'$, then, by 
Property~\ref{prp:2_planar_cross_twice}  there 
exist two (non identical) edges $(u,u')$ and $(v,v')$ that cross $(w,w')$ at $c$ 
and $c'$ respectively. Property~\ref{prp:2_planar_quasi} implies that 
vertices $u$ and $v$ always form a parallel pair, and so do vertices $u'$ and 
$v'$. 
%Since this property will be heavily used throughout the paper, we shall not explicitly state it every time, but rather implicitly imply it.

%\todo[inline]{add in every lemma the assumption about edges crossing twice}

\begin{lemma}
Let $\Gamma(G)$ be a PMCM $2$-planar drawing of an optimal $2$-planar graph $G$. Any edge that is crossed twice in  $\Gamma(G)$ is a chord of a true-planar $5$-cycle in $\Gamma(G)$. 
\label{lem:2_planar_small_faces}
\end{lemma}

\begin{proof}
Let $(w,w')$ be an edge of $G$ that is crossed twice in $\Gamma(G)$ and let 
$(u,u')$ and $(v,v')$ be the corresponding edges crossed by $(w,w')$.\todo{fix 
the notation of edges uniformly} This crossing configuration defines four 
corner pairs: \todo{fix a notation for corner and parallel pairs}$\langle w,u \rangle$, 
$\{w,u'\}$, $\{w,v\}$, $\{w,v'\}$, and two parallel pairs: $\{u,v\}$ and 
$\{u',v'\}$\todo{a ref to a fig is needed for the placement of crossings. Are 
there other x-ing configurations?}. By Properties~\ref{prp:corner} and 
~\ref{prp:parallel}, all corner edges are \pes, and either one or two of the 
parallel edges are \pes. We distinguish two cases depending on whether both 
parallel edges are \pes or not.

So suppose first, that both parallel edges $[u,v]$ and $[u',v']$ are \pes, as 
in Figure~\ref{fig:2_planar_both_parallel_before}. Then vertices 
$w,u,v,w',v',u'$ define a \pp $\mathcal{C}$ of six vertices (grey shaded in 
Figure~\ref{fig:2_planar_both_parallel_before}). There exist at most five edges 
of $\Gamma(G)$ that pass through the interior of $\mathcal{C}$: edges $(w,w')$, 
$(u,u')$, $(v,v')$ and at most two other edges that cross $(u,u')$ or $(v,v')$. 
%Note that since the \pr is an open topological region without vertices in its interior, there can not be an edge that lies entirely in its interior. 
We proceed by removing  edges $(w,w')$, $(u,u')$, $(v,v')$ and all other edges 
that cross $(u,u')$ or $(v,v')$, and replace them with the $2$-planar pattern of 
Figure~\ref{fig:2_planar_both_parallel_after}\todo{use 
Lemma~\ref{lem:exchange}}. In the derived graph there exist $6$ edges drawn 
inside $\mathcal{C}$ and do not cross its boundary. Hence, the derived graph has 
more edges than $G$; a contradiction to the optimality of $G$. \todo{this 
sentence is not needed. It should be obvious so far!}Note that even in the case 
where $\mathcal{C}$ is not simple, i.e. the vertices that define its boundary 
are not all distinct, the above argument still holds, as can be seen for example 
in Figure~\ref{fig:2_planar_both_parallel_non_simple}, where $w=u$ and $v=v'$.
\todo{fix the figure}

%So suppose first, that both parallel edges $(u,v)$ and $(u',v')$ are \pes, as in Figure~\ref{fig:2_planar_both_parallel_before}. Then vertices $w,u,v,w',v',u'$ define a \pp $\mathcal{C}_6$ of six vertices (grey shaded in Figure~\ref{fig:2_planar_both_parallel_before}). The \pr $\mathcal{R}_6$ of $\mathcal{P}_6$ contains no vertices in its interior, and at most five edges of $\Gamma(G)$ pass through $\mathcal{R}_6$: edges $(w,w')$, $(u,u')$, $(v,v')$ and at most two other edges that cross $(u,u')$ or $(v,v')$. Note that since the \pr is an open topological region without vertices in its interior, there can not be an edge that lies entirely in its interior. We proceed by removing  edges $(w,w')$, $(u,u')$, $(v,v')$ and all other edges that cross $(u,u')$ or $(v,v')$, and replace them with the $2$-planar pattern of Figure~\ref{fig:2_planar_both_parallel_after}. In the derived graph there exist $6$ edges drawn in $\mathcal{R}_6$ and do not cross the boundary of the \pp. Hence, the derived graph has more edges than $G$; a contradiction to the optimality of $G$. Note that even in the case where the \pp is not simple, i.e. the vertices that define its boundary are not all distinct, the above argument still holds, as can be seen for example in Figure~\ref{fig:2_planar_both_parallel_non_simple}, where $w=u$ and $v=v'$.
%\todo{fix the figure}

Suppose now that parallel edge $[u,v]$ is not a \pe. Then, it is $u=v$ and 
$[u,v]$ is an homotopic self-loop; refer to 
Figure~\ref{fig:2_planar_one_parallel_before}. This time, vertices 
$w,u,w',v',u'$ define a \pp $\mathcal{C_5}$ of five vertices (grey shaded in 
Figure~\ref{fig:2_planar_one_parallel_before}). As in the previous case, at most 
five edges of $\Gamma(G)$ pass through the interior of $\mathcal{C}_5$. We 
proceed by removing  edges $(w,w')$, $(u,u')$, $(v,v')$ and all other edges that 
cross $(u,u')$ or $(v,v')$, and replace them with the $2$-planar pattern of 
Figure~\ref{fig:2_planar_one_parallel_after}. In the derived graph there exist 
$5$ edges drawn inside $\mathcal{C}_5$ plus the five \pes of its boundary. Since 
$G$ is optimal, it follows that:
\begin{enumerate}
\item there exist exactly two edges other than $(w,w')$, say $e$ and $e'$, that cross $(u,v)$ and $(u',v')$ respectively, and
\item  the \pes of the boundary of  $\mathcal{C}_5$ already exist in the drawing $\Gamma(G)$, i.e. $\mathcal{C}_5$ is a cycle of length $5$. 
\end{enumerate}

%Suppose now that parallel edge $(u,v)$ is not a \pe. Then, it is $u=v$; refer to Figure~\ref{fig:2_planar_one_parallel_before}. This time, vertices $w,u,w',v',u'$ define a \pp $\mathcal{P}_5$ of five vertices (grey shaded in Figure~\ref{fig:2_planar_one_parallel_before}). As in the previous case, at most five edges of $\Gamma(G)$ pass through the polygonal region $\mathcal{R}_5$. We proceed by removing  edges $(w,w')$, $(u,u')$, $(v,v')$ and all other edges that cross $(u,u')$ or $(v,v')$, and replace them with the $2$-planar pattern of Figure~\ref{fig:2_planar_one_parallel_after}. In the derived graph there exist $5$ edges drawn in $\mathcal{R}_5$ plus the five \pes of the boundary of $\mathcal{P}_5$. Since $G$ is optimal, it follows that:
%\begin{enumerate}
%\item there exist exactly two edges other than $(w,w')$, say $e_1$ and $e_2$, that cross $(u,v)$ and $(u',v')$ respectively, and
%\item  the \pes of the boundary of  $\mathcal{P}_5$ already exist in the drawing $\Gamma(G)$, i.e. $\mathcal{P}_5$ is a cycle, say $C$, of length $5$. 
%\end{enumerate}

If $\mathcal{C}_5$ is a true-planar $5$-cycle in $\Gamma(G)$ the lemma holds, so 
suppose that this is not the case. Then, at least one of edges $e$ or $e'$ 
crosses a boundary edge of $\mathcal{C}_5$. Suppose w.l.o.g. that  edge 
$e=(x,x')$ crosses edge $(v,v')$ of $\mathcal{C}_5$ at crossing point $c$ (the 
case where $e$ crosses edge $(u,v)$ can be treated similarly);\todo{is it clear 
that it can't cross another edge of C?} refer to 
Figure~\ref{fig:2_planar_one_parallel_extra}. Note that $e$ already has two 
crossings in the drawing $\Gamma(G)$. Then, at least one of the edge segments 
$(x,c)$ or $(c,x')$ of $e$ does not pass through the interior of 
$\mathcal{C}_5$. Suppose w.l.o.g. that this is edge segment $(x,c)$  of $e$. 
Then, vertices \todo{use uniform notation}$x$ - $v$, and $x$ - $v'$ define two 
corner pairs of vertices. Hence vertices $w,u,w',v',x,v$ define a \pp 
$\mathcal{C}_6$ on six vertices, with exactly six edges passing through its 
interior. We remove all edges that pass through $\mathcal{C}_6$ and replace it 
with the $2$-planar pattern of Figure~\ref{fig:2_planar_one_parallel_final}: we 
add one vertex and a total of $12$ edges\todo{use Lemma~\ref{lem:exchange}}. 
The derived graph, say $G'$, is $2$-planar and has $n'=n+1$ vertices and 
$m'=m+6$ edges (where $n$, $m$ are the number of vertices and edges of $G$ 
respectively). Hence $m'=5n'-9>5n'-10$, i.e. $G'$ has more edges than allowed; a 
contradiction.


%If $C$ is a true-planar $5$-cycle in $\Gamma(G)$ the lemma holds, so suppose that this is not the case. Then, at least one of edges $e_1$ or $e_2$ crosses $C$. Suppose w.l.o.g. that  edge $e_1=(w_1,w'_1)$ crosses edge $(v,v')$ of the $5$-cycle $C$ at crossing point $c$ (the case where $e_1$ crosses edge $(u,v)$ can be treated similarly);\todo{is it clear that it can't cross another edge of C?} refer to Figure~\ref{fig:2_planar_one_parallel_extra}. Note that $e_1$ already has two crossings in the drawing $\Gamma(G)$. Then, at least one of the edge segments $(w_1,c)$ or $(c,w'_1)$ of $e_1$ does not pass through the \pr $\mathcal{R}_5$. Suppose w.l.o.g. that this is edge segment $(w_1,c)$  of $e_1$. Then, vertices $w_1$ - $v$, and $w_1$ - $v'$ define two corner pairs of vertices. Hence vertices $w,u,w',v',w_1,v$ define a \pp $\mathcal{P}_6$ on six vertices, with exactly six edges passing through its \pr, say $\mathcal{R}_6$. We remove all edges that pass through $\mathcal{R}_6$ and replace it with the $2$-planar pattern of Figure~\ref{fig:2_planar_one_parallel_final}: we add one vertex and a total of $12$ edges. The derived graph, say $G'$, is $2$-planar and has $n'=n+1$ vertices and $m'=m+6$ edges (where $n$, $m$ are the number of vertices and edges of $G$ respectively). Hence $m'=5n'-9>5n'-10$, i.e. $G'$ has more edges than allowed; a contradiction.


\end{proof}

\begin{figure}[htb]
    \centering
    \begin{minipage}[b]{.24\textwidth}
        \centering
        \includegraphics[width=\textwidth,page=1]{images/2_planar_potential_parallel}
        \subcaption{~}\label{fig:2_planar_both_parallel_before}
    \end{minipage}
    \begin{minipage}[b]{.24\textwidth}
        \centering
        \includegraphics[width=\textwidth,page=2]{images/2_planar_potential_parallel}
        \subcaption{~}\label{fig:2_planar_both_parallel_after}
    \end{minipage}
	\begin{minipage}[b]{.24\textwidth}
        \centering        
        \includegraphics[width=\textwidth,page=3]{images/2_planar_potential_parallel}
        \subcaption{~}\label{fig:2_planar_both_parallel_non_simple}
    \end{minipage}
		
    \begin{minipage}[b]{.24\textwidth}
        \centering
        \includegraphics[width=\textwidth,page=4]{images/2_planar_potential_parallel}
        \subcaption{~}\label{fig:2_planar_one_parallel_before}
    \end{minipage}
		\begin{minipage}[b]{.24\textwidth}
        \centering
        \includegraphics[width=\textwidth,page=5]{images/2_planar_potential_parallel}
        \subcaption{~}\label{fig:2_planar_one_parallel_after}
    \end{minipage}
		\begin{minipage}[b]{.24\textwidth}
        \centering
        \includegraphics[width=\textwidth,page=6]{images/2_planar_potential_parallel}
        \subcaption{~}\label{fig:2_planar_one_parallel_extra}
    \end{minipage}
		\begin{minipage}[b]{.24\textwidth}
        \centering
        \includegraphics[width=\textwidth,page=7]{images/2_planar_potential_parallel}
        \subcaption{~}\label{fig:2_planar_one_parallel_final}
    \end{minipage}
    \caption{%
    Different configurations used in Lemma~\ref{lem:2_planar_small_faces}.}
    \label{fig:2_planar_potential_parallel}
\end{figure}




By Lemma~\ref{lem:2_planar_small_faces}, we have that any edge of $G$ that is 
crossed twice in a PMCM-drawing $\Gamma(G)$ is a chord of a true-planar 
$5$-cycle. So, it remains to consider edges of $G$ that have only one crossing 
in $\Gamma(G)$. In fact, the following lemma states that there there are 
no such edges in the drawing.

\begin{lemma}\label{lem:2_planar_one_crossing}
Let $\Gamma(G)$ be a PMCM $2$-planar drawing of an optimal $2$-planar graph $G$. Then every edge of $\Gamma(G)$ is either true-planar or has exactly two crossings.
\end{lemma}
\begin{proof}
By the proof of Lemma~\ref{lem:2_planar_small_faces}, we have that, for any edge 
of $G$ that is crossed twice in $\Gamma(G)$, both edges that cross this 
particular edge also have two crossing in $\Gamma(G)$. In other words, the 
crossing component of an edge with two crossings, contains only edges with two 
crossings. This implies that if edges $(u,u')$ and $(v,v')$ cross in $\Gamma(G)$ 
and $(u,u')$ has only one crossing, then the same holds for $(v,v')$. \todo{use 
crossing component}So, we have four corner pairs of  vertices: \todo{use 
uniform notation}$u$ - $v$, $u$ - $v'$, $u'$ - $v$, $u'$ - $v'$, and all corner 
edges are \pes; refer to Figure~\ref{fig:2_planar_one_crossing_before}. Vertices 
 $u,v,v',u'$ define a \pp $\mathcal{C}$ of four vertices (grey shaded in 
Figure~\ref{fig:2_planar_one_crossing_before}). Since edges $(u,u')$ and 
$(v,v')$ have only one crossing each, the \pes of the boundary of $\mathcal{C}$ 
exist in $\Gamma(G)$ and are true-planar edges (Indeed if any of these \pes does 
not exist, one could add them in $\Gamma(G)$ without introducing any crossings 
or creating homotopic edges, contradicting the optimality of $G$). We proceed by 
removing  edges $(u,u')$ and $(v,v')$, and replace them with the $2$-planar 
pattern of Figure~\ref{fig:2_planar_one_crossing_after}. The derived graph, say 
$G'$ is $2$-planar and has $n'=n+2$ vertices and $m'=m+11$ edges (where $n$, $m$ 
are the number of vertices and edges of $G$ respectively). Hence 
$m'=5n'-9>5n'-10$, i.e. $G'$ has more edges than allowed; a contradiction.
%By the proof of Lemma~\ref{lem:2_planar_small_faces}, we have that, for any edge of $G$ that is crossed twice in $\Gamma(G)$, both edges that cross this particular edge also have two crossing in $\Gamma(G)$. This implies that if edges $(u,u')$ and $(v,v')$ cross in $\Gamma(G)$ and $(u,u')$ has only one crossing, then the same holds for $(v,v')$. \todo{use propagated set}So, we have four corner pairs of  vertices: $u$ - $v$, $u$ - $v'$, $u'$ - $v$, $u'$ - $v'$, and all corner edges are \pes; refer to Figure~\ref{fig:2_planar_one_crossing_before}. Vertices  $u,v,v',u'$ define a \pp $\mathcal{P}_4$ of four vertices (grey shaded in Figure~\ref{fig:2_planar_one_crossing_before}). Since edges $(u,u')$ and $(v,v')$ have only one crossing each, the \pes of the boundary of $\mathcal{P}_4$ exist in $\Gamma(G)$ and are true-planar edges (Indeed if any of these \pes does not exist, one could add them in $\Gamma(G)$ without introducing any crossings or creating homotopic edges, contradicting the optimality of $G$). We proceed by removing  edges $(u,u')$ and $(v,v')$, and replace them with the $2$-planar pattern of Figure~\ref{fig:2_planar_one_crossing_after}. The derived graph, say $G'$ is $2$-planar and has $n'=n+2$ vertices and $m'=m+11$ edges (where $n$, $m$ are the number of vertices and edges of $G$ respectively). Hence $m'=5n'-9>5n'-10$, i.e. $G'$ has more edges than allowed; a contradiction.

%By the proof of Lemma~\ref{lem:2_planar_small_faces}, we have that, for any edge of $G$ that is crossed twice in $\Gamma(G)$, both edges that cross this particular edge also have two crossing in $\Gamma(G)$. This implies that if edges $(u_1,v_1)$ and $(u_2,v_2)$ cross in $\Gamma(G)$ and $(u_1,v_1)$ has only one crossing, then the same holds for $(u_2,v_2)$. So, we have four corner pairs of  vertices: $u_1$ - $u_2$, $u_1$ - $v_2$, $v_1$ - $u_2$, $v_1$ - $v_2$, and all the potential corner edges exist; refer to Figure~\ref{fig:2_planar_one_crossing_before}. Vertices  $u_1,u_2,v_2,v_1$ define a \pp $\mathcal{P}_4$ of four vertices (grey shaded in Figure~\ref{fig:2_planar_one_crossing_before}). Since edges $(u_1,v_1)$ and $(u_2,v_2)$ have only one crossing each, the \pes of the boundary of $\mathcal{P}_4$ exist in $\Gamma(G)$ and are true-planar edges (Indeed if any of these \pes does not exist, one could add them in $\Gamma(G)$ without introducing any crossings or creating homotopic edges, contradicting the optimality of $G$). We proceed by removing  edges $(u_1,v_1)$ and $(u_2,v_2)$, and replace them with the $2$-planar pattern of Figure~\ref{fig:2_planar_one_crossing_after}. The derived graph, say $G'$ is $2$-planar and has $n'=n+2$ vertices and $m'=m+11$ edges (where $n$, $m$ are the number of vertices and edges of $G$ respectively). Hence $m'=5n'-9>5n'-10$, i.e. $G'$ has more edges than allowed; a contradiction.\qed
\end{proof}


\begin{figure}[htb]
    \centering
    \begin{minipage}[b]{.24\textwidth}
        \centering
        \includegraphics[width=\textwidth,page=1]{images/2planar_one_crossing}
        \subcaption{~}\label{fig:2_planar_one_crossing_before}
    \end{minipage}
    \begin{minipage}[b]{.24\textwidth}
        \centering
        \includegraphics[width=\textwidth,page=2]{images/2planar_one_crossing}
        \subcaption{~}\label{fig:2_planar_one_crossing_after}
    \end{minipage}
    \caption{%
    Configurations used in Lemma~\ref{lem:2_planar_one_crossing}}.
    \label{fig:2_planar_one_crossing}
\end{figure}


 
Now we are ready to prove the main property of every  PMCM-drawing $\Gamma(G)$ of a maximal $2$-planar graph. By Lemmas~\ref{lem:2_planar_small_faces} and \ref{lem:2_planar_one_crossing} we have that there exist only edges with two crossings and true-planar edges in $\Gamma(G)$ that create only true-planar $5$-cycles. 
We claim that the true planar skeleton $\Pi(G)$ is connected. Suppose that this is not the case, and that $\Pi(G)$ is not connected. Since $G$ is connected, there exist two connected components $\Pi(G_1)$ and $\Pi(G_2)$ of $\Pi(G)$ and an edge $(u_1,u_2)$ in $G$ such that $u_1\in V(G_1)$ and $u_2\in V(G_2)$. Clearly  $(u_1,u_2)$ is not true-planar, but by Lemma~\ref{lem:2_planar_small_faces} there exists a true-planar $5$-cycle with $(u_1,u_2)$ drawn as a chord. This implies that there exists a path of true-planar edges of $\Pi(G)$ connecting vertices $u_1$ and $u_2$; a contradiction. Therefore, we have proven the following:
 

 \begin{corollary}\label{cor:2_planar_faces}
  The true planar skeleton $\Pi(G)$ of a PMCM-drawing $\Gamma(G)$ of a maximal $2$-planar graph $G$ contains only faces of length $5$.
 \end{corollary}



%Recall that at the beginning of this section we made the assumption that there is no pair of edges that cross twice in the drawing $\Gamma(G)$. In the following, we prove that in a planar-maximal crossing-minimal drawing there are no such pairs of edges.



%\begin{lemma}
%Let $\Gamma(G)$ be a $2$-planar drawing of an optimal $2$-planar graph $G$ on $n$ vertices. There is no pair of edges that cross twice in $\Gamma(G)$. 
%\label{lem:2_planar_cross_twice}
%\end{lemma}
%
%\begin{proof}
%Suppose that there exist edges $(u_1,v_1)$ and $(u_2,v_2)$ that cross twice at crossing points $c_1$ and $c_2$. Consider the bounded region $R$ defined by $(c_1,c_2)$ segment of edges $(u_1,v_1)$ and $(u_2,v_2)$. Since edges $(u_1,v_1)$ and $(u_2,v_2)$ both have two crossings, no other edge of $G$ crosses the boundary of $R$. Hence, if there exists at least one vertex of $G$ inside this region, $G$ is not connected; a contradiction.\todo{mention that G is connected} On the other hand, if $R$ is empty, we could redraw edges $(u_1,v_1)$ and $(u_2,v_2)$ so that they do not cross; a contradiction to the fact that $\Gamma(G)$ is crossing-minimal.\qed
%\end{proof}

%By combining Corollary~\ref{cor:2_planar_faces} and Property~\ref{prp:2_planar_cross_twice}, we can characterize all optimal $2$-planar graphs:


 %By Property~\ref{prp:2_planar_cross_twice} we have that Corollary~\ref{cor:2_planar_faces} holds for all maximal $2$-planar graphs. 
Since the true planar skeleton of such a graph contains only faces of length $5$, we can start with a $5$-tiling of the plane\todo{rephrase}. Now in the interior of every face of length $5$, we can add all  $5$ missing edges using the $2$-planar pattern of Figure~\ref{fig:2_planar_one_parallel_after}.


% ============================================================================
\section{Characterization of 3-planar graphs}
\label{sec:3planar}
% ============================================================================

Let $G$ be an optimal $3$-planar graph on $n$ vertices (and therefore with $5.5n-11$ edges) and let $\Gamma(G)$ be a PMCM-drawing of $G$, i.e. $\Gamma(G)$ has the maximum number of true-planar edges among all potential $3$-planar drawing of $G$ and, subject to this restriction, $\Gamma(G)$ has also the minimum number of crossings. In the following, we examine structural properties of $\Gamma(G)$ and in particular we show that the true-planar skeleton $\Pi(G)$ of $\Gamma(G)$ consists of faces of length six. 


\begin{lemma}\label{lem:no-of-edges}
A facial $6$-cycle $f$ of the true-planar skeleton $\Pi(G)$ of a PMCM-drawing $\Gamma(G)$ of an optimal $3$-planar graph $G$ $f$ cannot contain more than $8$ chords drawn completely in its interior.
\end{lemma}
\begin{proof}
If $f$ contains more than $8$ chords drawn completely in its interior, then one can observe that at least one chord of $f$ has $4$ crossings, since the boundary edges of $f$ and its chords would form a complete graph on six vertices. On the other hand, Figure~\ref{fig:6gon} is a certificate that $f$ can indeed contain $8$ chords without violating $3$-planarity. This completes the proof of this lemma.
\end{proof}

In the following, we consider a \pp $\mathcal{C}$ on $6$ vertices in $\Gamma(G)$. Let $E_{\mathcal{C}}$ be the set of edge-segments that pass through the interior of $\mathcal{C}_6$. Note that $E_{\mathcal{C}}$ may contain also edges that are chords of $\mathcal{C}_6$. We call $E_{\mathcal{C}}$ the \emph{passing-through segments} of $\mathcal{C}_6$. 


\begin{lemma}
Let $\Gamma(G)$ be a PMCM-drawing of an optimal $3$-planar graph $G$, and suppose that there exists a \pp $\mathcal{C}$ of $6$ vertices in $\Gamma(G)$, such that the potential boundary edges of $\mathcal{C}$ exist in $\Gamma(G)$. Let $E_{\mathcal{C}}$ be the set of passing-through segments of $\mathcal{C}$. If the following conditions hold, then $\mathcal{C}$ is an empty true-planar $6$-cycle in $\Gamma(G)$ and all edges with edge-segments in $E_{\mathcal{C}}$ are drawn as chords in its interior.
\begin{enumerate}[C.1:]
\item \label{cnd:1} $|E_{\mathcal{C}}|\leq 8$, and, 
\item \label{cnd:2} every edge-segment of $E_{\mathcal{C}}$ has at least one crossing in the interior of $\mathcal{C}$.
\end{enumerate}
\label{lem:size9}
\end{lemma}
\begin{proof}
We start with the following observation: If $e$ be an edge of $G$, then due to $3$-planarity at most one edge-segment of $e$ belongs to $E_{\mathcal{C}}$. More precisely, if $E_{\mathcal{C}}$ contains at least two edge-segments of $e$, then we claim that $e$ has at least four crossings. By Condition C.\ref{cnd:2} each of the two edge-segments of $e$ contributes one crossing to $e$. Since $\mathcal{C}$ is empty and contains two edge-segments of $e$, it follows that $e$ enters and exits $\mathcal{C}$. Hence, $e$ has two more crossings, summing up to a total of at least four crossings. 

Let $v_1,v_2,\dots,v_6$ be the vertices of $\mathcal{C}$. If all edges with edge-segments in $E_{\mathcal{C}}$ completely lie in $\mathcal{C}$,  then $\mathcal{C}$ is a true-planar $6$-cycle and the lemma trivially holds. Otherwise, there is at least one edge $e$ with an edge-segment in $E_{\mathcal{C}}$, that crosses an edge of $\mathcal{C}$. W.l.o.g.~we can assume that $e$ crosses edge $(v_1,v_6)$ of $\mathcal{C}$ at point $c$ (refer to Figure~\ref{fig:cs1}). If $w$ and $w'$ are the two endpoints of $e$, then by the observation we made at the beginning of the proof it follows that either edge-segment $(w,c)$ or edge-segment $(c,w')$ of $e$ lies entirely drawn outside $\mathcal{C}$ (as otherwise $e$ would have at least two edge-segments in $E_{\mathcal{C}}$). W.l.o.g.~assume that edge-segment $(w,c)$ of $e$ is drawn outside $\mathcal{C}$. Then, corner edges $[v_1,w]$ and $[w,v_6]$ are \pes (by Property~\ref{prp:corner}). 

\begin{figure}[t!]
    \centering
    \begin{minipage}[b]{.16\textwidth}
        \centering
        \includegraphics[width=\textwidth,page=1]{images/polygon_conf}
        \subcaption{~}\label{fig:6gon}
    \end{minipage}
    \begin{minipage}[b]{.16\textwidth}
        \centering
        \includegraphics[width=\textwidth,page=1]{images/3planar_polygon}
        \subcaption{~}\label{fig:cs1}
    \end{minipage}
    \begin{minipage}[b]{.16\textwidth}
        \centering
        \includegraphics[width=\textwidth,page=2]{images/3planar_polygon}
        \subcaption{~}\label{fig:cs2}
    \end{minipage}
	\begin{minipage}[b]{.16\textwidth}
        \centering
        \includegraphics[width=\textwidth,page=3]{images/3planar_polygon}
        \subcaption{~}\label{fig:7_stick}
    \end{minipage}
    \begin{minipage}[b]{.16\textwidth}
        \centering
        \includegraphics[width=\textwidth,page=4]{images/3planar_polygon}
        \subcaption{~}\label{fig:cs_final}
    \end{minipage}
    \begin{minipage}[b]{.16\textwidth}
        \centering
        \includegraphics[width=\textwidth,page=1]{images/3planar_independent}
        \subcaption{~}\label{fig:independent}
    \end{minipage}
    \caption{%
    Different configurations used in  
    (a)~Lemma~\ref{lem:no-of-edges},
    (b)-(e)~Lemma~\ref{lem:size9}, 
    (f)~Lemma~\ref{lem:3_planar_independent}.}
    \label{fig:replacements_2}
\end{figure}

Recall that edge $e$ has one crossing in the interior of $\mathcal{C}$ (by Condition C.\ref{cnd:2} of the lemma) and one more crossing with edge $(v_1,v_6)$. By $3$-planarity, it follows that edge $e$ may have at most one more crossing, say with edge $e'$. Note that $e'$ may have an edge-segment in $E_{\mathcal{C}}$. Vertices $w,v_1,v_2,\dots,v_6$ define a \pp $\mathcal{C}'$ on $7$ vertices (see Figure~\ref{fig:cs2}). The set of passing-through segments $E_{\mathcal{C}'}$ of $\mathcal{C}'$ contains all edge-segments of $E_{\mathcal{C}}$ (that is, $E_{\mathcal{C}} \subseteq E_{\mathcal{C}'}$) plus at most two additional edge-segments: the one defined by edge $(v_1,v_6)$, and possibly an edge-segment of $e'$. Hence $|E_{\mathcal{C}'}| \leq 10$. 

\begin{claim}
All edges with an edge-segment in $E_{\mathcal{C}'}$ have at least one crossing in the interior of $\mathcal{C}'$.
\label{nclm:1}
\end{claim}
\begin{proof}
The claim clearly holds for all edge-segments of $E_{\mathcal{C}}$ (recall that $E_{\mathcal{C}}\subset E_{\mathcal{C}'}$). Since $(v_1,v_{6})$ and $e'$ both cross $e$ in the interior of $\mathcal{C}'$, it follows that the remaining edge-segments of $\mathcal{C}'$ (i.e., the ones defined by edges $(v_1,v_{6})$ and $e'$) have at least one crossing in the interior of $\mathcal{C}'$.
\end{proof}

\begin{claim}
At least one edge with an edge-segment in $E_{\mathcal{C}'}$ crosses one of the edges of $\mathcal{C}'$.
\label{nclm:2}
\end{claim}
\begin{proof}
Assume to the contrary that all edges with an edge-segment in $E_{\mathcal{C}'}$ do not cross $\mathcal{C}'$. Then, all edges with an edge-segment in $E_{\mathcal{C}'}$ can be drawn completely in the interior of $\mathcal{C}'$, which implies that all \pes of $\mathcal{C}'$ can be added in $\Gamma(G)$ (if they are not present already). This, however, contradicts Property~\ref{prp:3planar_odd_cycle}. 
\end{proof}

\noindent By Claim~\ref{nclm:2}, it follows that there exists an edge, say $g$, that crosses an edge of $\mathcal{C}'$. W.l.o.g.~assume that $g$ crosses $[w,v_1]$ of $\mathcal{C}'$.

\begin{claim}
All boundary edges of $\mathcal{C}'$ exist in $\Gamma(G)$
\label{nclm:3}
\end{claim}
\begin{proof}
In order to prove this claim, we remove from the interior of  $\mathcal{C}'$ all edges with an edge-segment in $E_{\mathcal{C}'}$ and replace them with the $10$ edges of the $3$-planar crossing pattern of Figure~\ref{fig:7_stick}. This allows us to add all boundary edges of  $\mathcal{C}'$ in $\Gamma(G)$ (if they are not present already). In addition, we can redraw the segment of $g$ in the interior of $\mathcal{C}'$ so that: % 
\begin{inparaenum}[(i)]
\item $g$ emanates from vertex $v_6$ of $\mathcal{C}'$,
\item $g$ crosses only \pe $[v_1,w]$ at point $c$, and
\item $g$ has no other crossings in the interior of  $\mathcal{C}'$.
\end{inparaenum}
Hence, $3$-planarity is preserved and the derived graph has at least as many edges as $G$. Since $G$ is optimal, it follows that all boundary edges of $\mathcal{C}'$ must exist in $\Gamma(G)$, which completes the proof of the claim.
\end{proof}

By Claim~\ref{nclm:3}, it follows that $g$ has one crossing in the interior of  $\mathcal{C}'$ and one more crossing with edge $(v_1,w)$ at point $c'$. We follow the same approach we used for expanding $\mathcal{C}$ (that has $6$ vertices) to $\mathcal{C}'$ (that has $7$ vertices). We can find an endpoint of $g$, say $z$, such that $w,z,v_1,v_2,\dots,v_6$ define a \pp $\mathcal{C}''$ on $8$ vertices. Furthermore, the set of passing-through segments $E_{\mathcal{C}''}$ of $\mathcal{C}''$ has at most $12$ elements (at most two more than $E_{\mathcal{C}'}$). We proceed by removing all edges with an edge-segment in $E_{\mathcal{C}''}$ and split $\mathcal{C}''$ into two true-planar cycles of length $6$ and $4$ respectively, by adding  true-planar chord $(v_1,v_6)$; refer to Figure~\ref{fig:cs_final}. In the interior of the $6$ cycle, we add the $8$ edges of the $3$-planar pattern of Figure~\ref{fig:6gon}. In the interior of the $4$-cycle, we add a vertex $x$ with a true planar edge $(v_1,x)$. Vertices $v_1,x,v_1,w,z,v_6$ define a new \pp on $6$ vertices, allowing us to add $8$ more edges in its interior. 

Summarizing the above, we removed at most $12$ edges, added a vertex and a total of $18$ edges. If $n$ and $m$ are the number of vertices and edges of $G$, then the derived graph $G'$ has $n'=n+1$ vertices and at least $m'\geq m+6$ edges. The last equation gives $m'\geq 5.5n'-10.5$, i.e. $G'$ has more edges than allowed; a contradiction.
\end{proof}

Let $(u,v)$ be an edge of $G$ that is crossed by two edges $(u_1,v_1)$ and $(u_2,v_2)$ in $\Gamma(G)$. By Property~\ref{prp:parallel} it follows that at least one of parallel edges $[u_1,u_2]$ and $[v_1,v_2]$ is a \pe of $\Gamma(G)$. Recall that in the case where both parallel edges $[u_1,u_2]$ and $[v_1,v_2]$ are \pes of $\Gamma(G)$, edges $(u_1,v_1)$ and $(u_2,v_2)$ are called independent.

\begin{lemma}\label{lem:3_planar_independent}
Let $\Gamma(G)$ be a PMCM $3$-planar drawing of an optimal $3$-planar graph~$G$. If edge $(u,v)$ is crossed by two independent edges $(u_1,v_1)$ and $(u_2,v_2)$, then $(u,v)$ is a chord of an empty true-planar $6$-cycle.
\end{lemma}
\begin{proof}
Refer to Figure~\ref{fig:independent}. Since $(u_1,v_1)$ and $(u_2,v_2)$ are independent edges, both parallel edges $[u_1,u_2]$ and $[v_1,v_2]$ are \pes. By Property~\ref{prp:corner}, it follows that corner edges $[u, u_1]$, $[u,v_1]$, $[u,u_2]$ and $[v,v_2]$ are also potential edges. Hence, vertices $u$, $u_1$, $u_2$, $v$, $v_2$ and $v_1$ define a \pp $\mathcal{C}$ on six vertices (gray-shaded in Figure~\ref{fig:independent}). Let $E$ be the set of passing-through edges of $\mathcal{C}$. We claim that $|E|\leq 8$. Indeed, set $E$ contains edges $(u,v)$, $(u_1,v_1)$, $(u_2,v_2)$, at most one other edge that crosses $(u,v)$, and at most four other edges that cross $(u_1,v_1)$ or $(u_2,v_2)$. We proceed by removing all edges of $E$ from $\Gamma(G)$ and replace them with the $3$-planar crossing pattern of Figure~\ref{fig:6gon}. In the derived drawing there exist $8$ edges that are drawn in the interior of $\mathcal{C}$ (without crossing $\mathcal{C}$). Since $G$ is optimal, it follows that $|E|=8$ and additionally all boundary edges of $\mathcal{C}$ exist in $\Gamma(G)$. Since $|E|=8$, all edges of $E$ have at least one crossing in the interior of $\mathcal{C}$. So, by Lemma~\ref{lem:size9} there exists an empty true-planar $6$-cycle that has $e$ as chord.
\end{proof}

We are now ready to state the main property of PMCM-drawings of optimal $3$-planar graphs. Recall that we denote by $\mathcal{X}(G)$ the crossing graph of $\Gamma(G)$ and by $\mathcal{X}(e)$ the crossing component of $\mathcal{X}(G)$ containing edge $e$ of $G$\todo{propaged set of crossings $\rightarrow$ crossing component, $S(e) \rightarrow \mathcal{X}(e)$, $C_G \rightarrow \mathcal{X}(G)$}.

\begin{lemma}\label{lem:3_planar_small_faces}
Let $\Gamma(G)$ be a PMCM $3$-planar drawing of an optimal $3$-planar graph $G$. Any edge that is crossed three times in  $\Gamma(G)$ is a chord of an \textcolor{blue}{empty true-planar $6$-cycle in $\Gamma(G)$.}%, where $6\leq s\leq 9$. 
\end{lemma}
\begin{proof}
%Recall that if $e$ is a chord of a true-planar $s$-cycle that has no vertices in its interior, then all edges of $\mathcal{X}(e)$ are also chords of this $s$-cycle

Let $e=(u,v)$ be an edge of $G$ that crosses with edges $e_i=(u_i,v_i)$ in $\Gamma(G)$, for $i=1,2,3$. Consider the crossing component $\mathcal{X}(e)$ where $e$ belongs to. We distinguish two cases depending on whether there exists an edge in $\mathcal{X}(e)$ that crosses with two independent edges or not.

Suppose that this is the case, and an edge of $\mathcal{X}(e)$ is crossed by two independent edges. Then by Lemma~\ref{lem:3_planar_independent}, there exists an empty true-planar $s$-cycle (without vertices in its interior) defining a \pp with this edge as a chord. Then all edges of $\mathcal{X}(e)$ are also drawn as chords of the $s$-cycle and the lemma follows.

Assume, now that all edges of $\mathcal{X}(e)$ with at least two crossings, are not crossed by independent edges in $\Gamma(G)$. Then, for edge $e$, that crosses with edges $e_1$, $e_2$, and $e_3$, we have that edges $e_i$, $e_j$ ($1\leq i<j\leq 3$) are not parallel edges. This implies that exactly one of parallel edges $[u_i,u_j]$ or $[v_i,v_j]$ is not a \pe. For the sake of simplicity let's refer to parallel edges $[u_i,u_j]$ and $[v_i,v_j]$ as the ``$u$-parallel edge'' and ``$v$-parallel edge'' respectively of $e_i$ and $e_j$. 
There are three ways to combine indices $i$ and $j$ and for every combination exactly one of the ``$u$-parallel edge'' and ``$v$-parallel edge'' is not a \pe. This implies that: there exist at least two combinations such that their ``$u$-parallel edges'' are not \pes, or there exist at least two combinations such that their ``$v$-parallel edges'' are not \pes of $\Gamma(G)$. W.l.o.g. assume that there exist two combinations of indices, say $i_1$ with $j_1$ and $i_2$ with $j_2$ for which their ``$u$-parallel edges'', namely $[u_{i_1},u_{j_1}]$ and $[u_{i_2},u_{j_2}]$  are not \pes. It is clear that since $i_1\neq j_1$, $i_2\neq j_2$ and $\left\{i_1,i_2,j_1,j_2\right\}\subseteq\left\{1,2,3\right\}$, at least two indices are the same. W.l.o.g. assume that $i_1=i_2$; other cases are symmetric. Let $i=i_1=i_2$. Then we have that parallel edges  $(u_i,u_{j_1})$ and $(u_i,u_{j_2})$ are not \pes, where $j_1\neq j_2$. This implies that $u_i=u_{j_1}$ and $u_i=u_{j_2}$. It is not hard to see that parallel edge $(u_{j_1},u_{j_2})$ can not be a \pe either; for an example see Figure~\ref{fig:3_planar_small_faces_example}, where $i=1$. This implies that for any combination of idices $i$ and $j$, parallel edge $[u_i,u_j]$ is not a \pe. Hence, for any edge of $\mathcal{X}(e)$ with three crossings in $\Gamma(G)$, we have the crossing pattern of Figure~\ref{fig:3_planar_small_faces_conf}, where the grey-shaded region has no vertices in its interior.

%Since there are three pairs of potential parallel edges $e_i$ and $e_j$, there exist at least two pairs, say $e_i$-$e_j$ and $e_{i'}$-$e_{j'}$ for which both \pes $(u_i,u_j)$ and $(u_{i'},u_{j'})$, or both \pes $(v_i,v_j)$ and $(v_{i'},v_{j'})$ do not exist. Assume w.l.o.g. that edges $(u_i,u_j)$ and $(u_{i'},u_{j'})$ do not exist. Since $1\leq i,i',j,j'\leq 3$ and $i\neq j$, $i'\neq j'$, we can also assume that $i=i'$. Then we have that potential parallel edges  $(u_i,u_j)$ and $(u_i,u_{j'})$ do not exist. This implies that $u_i=u_j$ and $u_i=u_j'$. Also, the potential parallel edge $(u_j,u_{j'})$ can not exist, since otherwise at least one of potential parallel edges $(u_i,u_j)$ or $(u_i,u_{j'})$ would also exist (for an example refer to Figure~\ref{fig:3_planar_small_faces_example}, where $i=1$). Hence, for any edge of $S(e)$ with three crossings in $\Gamma(G)$, we have the crossing pattern of Figure~\ref{fig:3_planar_small_faces_conf}, where the grey-shaded region has no vertices in its interior.

So far, vertices $u,v_1,v_2,v_3,v,u_1$ define a \pp $\mathcal{C}_6$ on six vertices. We want to apply Lemma~\ref{lem:size9} for $s=6$, and to achieve this, we claim that there exist at most $8$ edges that pass through the interior of $\mathcal{C}_6$ and satisfy Conditions C.1 and C.2 of the lemma, and also that all boundary edges of $\mathcal{C}_6$ belong in $\Gamma(G)$. We claim that any edge crossing $e_2$ must also cross $e_1$ or $e_3$. In order to prove the claim, suppose that there exists an edge $e'=(u',v')$ that crosses with $e_2$ in the interior of $\mathcal{C}_6$. Since $e_2\in \mathcal{X}(e)$, $e_2$ is not crossed by independent edges. So, edges $e'=(u',v')$ and $e=(u,v)$ are not independent edges, and exactly one of parallel edges $[u,u']$ or $[v,v']$ is not a \pe. Assume w.l.o.g. that parallel edge $[u,u']$ is not a \pe. This implies that $u=u'$ and that the region $R$ \todo{use the notation of preliminaries}defined by edges $e_2$, $e$ and $e'$ has no vertices in its interior (see Figure~\ref{fig:3_planar_small_faces_conf_middle_a}). Then, $e'$ must cross with $e_1$, as otherwise vertex $v_1$ would be in the interior of $R$; see Figure~\ref{fig:3_planar_small_faces_conf_middle_b}. Hence, any edge that crosses with $e_2$, must also cross with $e_1$ or $e_3$ as claimed. 

The above claim assures that there exist at most four other edges that pass through the interior of $\mathcal{C}_6$ and cross with edges $e_1$, $e_2$ or $e_3$, i.e. we have at most $8$ edges that pass through the interior of $\mathcal{C}_6$. We proceed by removing edges $e$, $e_i$ and all edges that cross with edges $e_i$ in  $\mathcal{C}_6$ ($i=1,2,3$), and replace them with the $3$-planar pattern of Figure~\ref{fig:6gon}. The derived graph has at least as many edges as $G$, and since $G$ is optimal, the boundary edges of $\mathcal{C}_6$ belong in the drawing $\Gamma(G)$. So, the vertices of $\mathcal{C}_6$ define an empty $6$-cycle in $\Gamma(G)$, and the set $E_{\mathcal{C}}$ of passing-through edge-segments of $\mathcal{C}_6$ has size $|E_{\mathcal{C}}|=8$ (Condition C.1). Furthermore, every edge-segment of $E_{\mathcal{C}}$ has at least one crossing in the interior of $\mathcal{C}_6$ (Condition C.2). By Lemma~\ref{lem:size9}, %for $s=6$, we have that $e$ is a chord of a true planar $s'$-cycle for $6\leq s'\leq 9$.
\textcolor{blue}{we have that $e$ is a chord of a true planar $6$-cycle.}
\end{proof}

\begin{figure}[htb]
    \centering
    \begin{minipage}[b]{.24\textwidth}
        \centering
        \includegraphics[width=\textwidth,page=1]{images/3planar_small_faces}
        \subcaption{~}\label{fig:3_planar_small_faces_example}
    \end{minipage}
    \begin{minipage}[b]{.24\textwidth}
        \centering
        \includegraphics[width=\textwidth,page=2]{images/3planar_small_faces}
        \subcaption{~}\label{fig:3_planar_small_faces_conf}
    \end{minipage}
    \begin{minipage}[b]{.24\textwidth}
        \centering
        \includegraphics[width=\textwidth,page=3]{images/3planar_small_faces}
        \subcaption{~}\label{fig:3_planar_small_faces_conf_middle_a}
    \end{minipage}
		\begin{minipage}[b]{.24\textwidth}
        \centering
        \includegraphics[width=\textwidth,page=4]{images/3planar_small_faces}
        \subcaption{~}\label{fig:3_planar_small_faces_conf_middle_b}
    \end{minipage}
    \caption{%
    (a):~If the parallel edge $[u_1,u_1]$ is a\pe for edges $(u_2,v_2)$ and $(u_3,v_3)$ then it is also a \pe for edges $(u_1,v_1)$ and $(u_3,v_3)$.  (b)-(d):~Configuration of Lemma~\ref{lem:3_planar_small_faces}.}
    \label{fig:3_planar_small_faces}
\end{figure}

By Lemma~\ref{lem:3_planar_small_faces}, we have that any edge of $G$ that is crossed three times in $\Gamma(G)$ is a chord of an empty true-planar %$s$-cycle for $6\leq s\leq9$. 
\textcolor{blue}{$6$-cycle}. So, it remains to consider edges of $G$ that have at most two crossings in $\Gamma(G)$, i.e. crossing components where every edge has at most two crossings. Note that we can not use directly  Lemma~\ref{lem:2_planar_small_faces}, since its proof is based on the fact that optimal $2$-planar graphs have at most $5n-10$ edges, however, we can formulate its \textcolor{red}{analogue lemma as follows:} \textcolor{blue}{proof and get the following result:}
\begin{lemma}
Let $\Gamma(G)$ be a PMCM $3$-planar drawing of an optimal $3$-planar graph $G$. Let $\mathcal{X}$ be a crossing component. \textcolor{blue}{Then there is at least one edge in $\mathcal{X}$ that has three crossings.}
\label{lem:3_planar_small_faces_2}
\end{lemma}
%\begin{lemma}
%Let $\Gamma(G)$ be a PMCM $3$-planar drawing of an optimal $3$-planar graph $G$. Let $\mathcal{X}$ be a crossing component where every edge has at most two crossings. Then edges of $\mathcal{X}$ are chords of a true-planar $5$-cycle in $\Gamma(G)$, that contains no vertices in its interior. 
%\label{lem:3_planar_small_faces_2}
%\end{lemma}
\begin{proof}
\textcolor{blue}{For a proof by contradiction, suppose that there exists a crossing component $\mathcal{X}$ where all edges have at most two crossings.}
We distinguish two cases depending on whether $\mathcal{X}$ contains an edge with two crossings or not. Suppose first that this is not the case. Then all edges of $\mathcal{X}$ have at most one crossing. Let $e\in \mathcal{X}$ crossing with $e'\in \mathcal{X}$. Clearly $\mathcal{X}=\left\{e,e'\right\}$. The four endpoints of edges $e$ and $e'$ define a \pp $\mathcal{C}_4$ on $4$ vertices, as in Figure~\ref{fig:3_planar_one_crossing_before} and there are no other edges passing through the interior of $\mathcal{C}_4$. We proceed by removing edges $e$ and $e'$ and replace them with the $3$-planar pattern of Figure~\ref{fig:3_planar_one_crossing_after}. The derived graph $G'$ has $n'=n+1$ vertices and $m'=m-2+8$ edges, where $n$ and $m$ are the number of vertices and edges of $G$ respectively. Then, $G'$ is $3$-planar and has $m'=5.5n'-10.5>5.5n'-11$ edges; a contradiction to the optimality of $G$.

Now, assume that there exists an edge $e\in \mathcal{X}$, where $e=(u,v)$ crossing with two edges $e_1=(u_1,v_1)$ and $e_2=(u_2,v_2)$. By Lemma~\ref{lem:3_planar_independent} edges $e_1$ and $e_2$ are not independent edges. \textcolor{blue}{Following the proof of Lemma~\ref{lem:2_planar_small_faces}, the endpoints of $(u,v)$, $(u',v')$ and $(u'',v'')$ define a \pp $\mathcal{C}$ on five vertices, with at most five edges passing through its interior. We can alter $G$, by redrawing the five edges passing through the interior of $\mathcal{C}$ so that $\mathcal{C}$ contains five chords and all its boundary edges are true-planar in the new drawing. Then the derived graph is optimal, since it has at least as many edges as $G$, but $\mathcal{C}$ is a true-planar $5$-cycle; a contradiction to Property~\ref{prp:3planar_even_order}.}


%Following the proof of Lemma~\ref{lem:2_planar_small_faces} we can find:
%\begin{enumerate}
%\item  a \pp on five vertices whose boundary edges define a true planar $5$-cycle with $e$ as a chord, or,
%\item a \pp $\mathcal{C}_6$ on six vertices with at most $6$ edges passing through its interior. In this case, we remove all edges with an edge-segment in the set of passing-through segments of $\mathcal{C}_6$ and use the $3$-planar pattern of Figure~\ref{fig:6gon} that gives $8$ edges in drawn in the interior of $\mathcal{C}_6$. The derived graph clearly has more edges than $G$, contradicting its optimality.
%\end{enumerate}
\end{proof}

\begin{figure}[htb]
    \centering
    \begin{minipage}[b]{.24\textwidth}
        \centering
        \includegraphics[width=\textwidth,page=1]{images/3planar_one_crossing}
        \subcaption{~}\label{fig:3_planar_one_crossing_before}
    \end{minipage}
    \begin{minipage}[b]{.24\textwidth}
        \centering
        \includegraphics[width=\textwidth,page=2]{images/3planar_one_crossing}
        \subcaption{~}\label{fig:3_planar_one_crossing_after}
    \end{minipage}
		%\begin{minipage}[b]{.24\textwidth}
        %\centering
        %\includegraphics[width=\textwidth,page=3]{images/3planar_one_crossing}
        %\subcaption{~}\label{fig:3_planar_triangle}
    %\end{minipage}
    \caption{%
    Configurations used in Lemma~\ref{lem:3_planar_small_faces_2}.}.
    \label{fig:3_planar_one_crossing_1}
\end{figure}

%\begin{lemma}
%Let $\Gamma(G)$ be a PMCM $3$-planar drawing of an optimal $3$-planar graph $G$. There is no true planar $3$-cycle without vertices in its interior. 
%\label{lem:3_planar_triangle}
%\end{lemma}
%\begin{proof}
%Suppose that there exists a true planar $3$-cycle in $\Gamma(G)$ without vertices in its interior. Then we can add a vertex in the interior of this cycle, and $6$ edges by using the $3$-planar pattern of Figure~\ref{fig:3_planar_triangle}. The derived graph $G'$ has $n'=n+1$ vertices and $m'=m+6$ edges, where $n$ and $m$ are the number of vertices and edges of $G$ respectively. Then, $G'$ is $3$-planar and has $m'=5.5n'-10.5>5.5n'-11$ edges; a contradiction.
%\end{proof}

%Now we are ready to prove the main property of every PMCM-drawing $\Gamma(G)$ of a maximal $3$-planar graph. 

Lemma~\ref{lem:3_planar_small_faces_2} states that any crossing component $\mathcal{X}$  contains at least one edge with three crossings, and by Lemma~\ref{lem:2_planar_small_faces} all edges of $\mathcal{X}$ are drawn as chords of a true-planar $6$-cycle. Also, similarly with the case of PMCM-drawings  of optimal $2$-planar graphs, one can prove that the true planar skeleton $\Pi(G)$ is connected. Therefore, we have proven the following:
%Lemma~\ref{lem:3_planar_small_faces_2} states that if in a crossing component $\mathcal{X}$ all edges have at most two crossings, then there exists a true-planar $5$-cycle in $\Gamma(G)$. However this contradicts Property~\ref{prp:3planar_odd_cycle}. Hence, $\mathcal{X}$ contains at least one edge with three crossings, and by Lemma~\ref{lem:2_planar_small_faces} all edges of $\mathcal{X}$ are drawn as chords of a true-planar $s$-cycle for $6\leq s\leq 9$. Again by Property~\ref{prp:3planar_odd_cycle} $s$ must be even, i.e. $s=6$ or $8$. Also, similarly with the case of PMCM-drawings  of optimal $2$-planar graphs, one can prove that the true planar skeleton $\Pi(G)$ is connected. Therefore, we have proven the following:
 

%By combining Lemmas~\ref{lem:3_planar_small_faces}, \ref{lem:3_planar_small_faces_2} and \ref{lem:3_planar_triangle} the following is straightforward:
 
%\begin{corollary}\label{cor:3_planar_faces_general}
%The true planar structure $\Pi(G)$ of a PMCM $3$-planar drawing of an optimal $3$-planar graph $G$ contains faces of length $6$ or $8$.
%\end{corollary}

\begin{corollary}\label{cor:3_planar_faces_general}
The true planar structure $\Pi(G)$ of a PMCM $3$-planar drawing of an optimal $3$-planar graph $G$ contains faces of length $6$.
\end{corollary}


 %\begin{corollary}\label{cor:3_planar_faces_general}
%The true planar structure $\Pi(G)$ of a PMCM $3$-planar drawing of an optimal $3$-planar graph $G$ on $n$ vertices contains faces of length at least $6$ and at most $9$.
%\end{corollary}
 %Now we are ready to prove the main property of  PMCM $3$-planar drawings of optimal $3$-planar graphs.
 %
 %\begin{lemma}\label{lem:3_planar_faces_final}
  %The true planar structure $\Pi(G)$ of a PMCM $3$-planar drawing of an optimal $3$-planar graph $G$ on $n$ vertices contains only faces of length $6$.
 %\end{lemma}
%
 %\begin{proof}
  %Let $m_{p}$ and $m_c$ be the total number of true planar and crossing edges of $G$ in any PMCM $3$-planar drawing $\Gamma(G)$. 
	%Clearly, $m_p+m_c=m$ where $m$ is the number of edges of $G$.
	%By Euler's formula we have that  
  %\begin{equation}
	%\label{eq:euler}
   %m_{p}+2=n+f\Rightarrow m_p-f=n-2
  %\end{equation}
  %%
	%where $f$ is the total number of faces of $\Pi(G)$. Let $f_s$ be the number of faces of $\Pi(G)$ of length $s$. By Corollary~\ref{cor:3_planar_faces_general} we have that $s=6$ or $s=8$. Hence $f=f_6+f_8$. 
	%Also, by counting the edges of all faces of $\Pi(G)$ we have that
  %\begin{equation}
	%\label{eq:faces}
   %2m_{p}=6f_6+8f_8\Rightarrow m_p=3f+f_8\Rightarrow m_p-f=2f+f_8\Rightarrow f=(m_p-f-f_8)/2
  %\end{equation}
%%
  %On the other hand for the number of crossing edges $m_c$, by Lemma~\ref{lem:no-of-edges} we obtain  
  %\begin{equation}
	%\label{eq:crossing}
   %m_{c}=8f_6+11f_8=8f+3f_8
  %\end{equation}
%%
  %Hence, for the total number of edges of $G$ we have 
	%%\todo{rewrite the equations}
  %
	%$\begin{array}{ll}
   %m& =m_{p}+m_{c}\stackrel{(\ref{eq:faces}), (\ref{eq:crossing})}{=}(3f+f_8)+(8f+3f_8)=11f+4f_8\\
		%& \stackrel{(\ref{eq:faces})}{=}11(m_p-f-f_8)/2+4f_8=11(m_p-f)/2-3f_8/2\\
    %& \stackrel{(\ref{eq:euler})}{=}11(n-2)/2-3f_8/2\leq 11(n-2)/2
  %\end{array}$
%%
  %%$\begin{array}{ll}
   %%|E(G)|= & 5.5n-11\\
    %%\Rightarrow &m_{p}+m_{c}= 5.5n-11\\
    %%\Rightarrow &m_{p}+m_{c}= 5.5(m_{p}-f)\\
    %%\Rightarrow &2m_{c}= 9m_{p}-11f\\
    %%\Rightarrow &10s_5+16s_6+18s_7+22s_8+28s_9 = 4.5(5s_5+6s_6+7s_7+8s_8+9s_9)\\
   %%&-11(s_5+s_6+s_7+s_8+s_9)\\
    %%\Rightarrow &10s_5+16s_6+18s_7+22s_8+28s_9=11s_5+16s_6+20.5s_7+25s_829.5s_9\\
    %%\Rightarrow &0= s_5+2.5s_7+3s_8+1.5s_9\\
   %%\Rightarrow &0=s_5=s_7=s_8=s_9
  %%\end{array}$
%Since $G$ is optimal $m=11(n-2)/2=5.5n-11$ must hold. 
%The last equality implies that $f_8=0$ and there are only faces of length $6$ in the true planar structure $\Pi(G)$.
 %\end{proof}

%Recall that at the beginning of this section we made the assumption that $\Gamma(G)$ is almost-simple, i.e. there is no pair of edges that cross twice in the drawing $\Gamma(G)$. In the following, we prove that in any PMCM drawing of an optimal $3$-planar graph $G$ on $n$ vertices there are no such pairs of edges.
 %By Lemma~\ref{lem:2_planar_faces} we can characterize all maximal $2$-planar graphs. Since the true planar subgraph of such a graph contains only faces of length $5$, we can start with a $5$-tiling of the plane. Now in the interior of every face of length $5$, we can add all  $5$ missing edges.



%By combining Lemma~\ref{lem:3_planar_faces_final} and Property~\ref{prp:3_planar_cross_twice}, we can characterize all optimal $3$-planar graphs:


 %By Property~\ref{prp:3_planar_cross_twice} we have that Lemma~\ref{lem:3_planar_faces_final} holds for all maximal $3$-planar graphs. 
Since the true planar subgraph of such a graph contains only faces of length $6$, we can start with a $6$-tiling of the plane\todo{rephrase}. Now in the interior of every face of length $6$, we can add all  $8$ missing edges using the $3$-planar pattern of Figure~\ref{fig:6gon}.
%



% ============================================================================
\section{Conclusions}
\label{sec:conclusions}
% ============================================================================

In this paper, we developed techniques that allow us to characterize optimal $2$- and $3$-planar graphs. In particular, we had to deal with non-homotopic parallel edges, loops and also crossing adjacent edges, which required special attention and efforts. Natural extensions of our results would be to characterize optimal $4$-planar graphs, optimal fan-planar and quasi-planar graphs. But it would also be interesting to have a characterization of the simple versions of these graph classes where (non-homotopic) parallel edges and loops are forbidden. Another interesting line of research is the recognition of optimal $2$- and $3$-planar graphs. We note here the recent result by Brandenburg~\cite{DBLP:journals/corr/Brandenburg16a} who gives a linear time algorithm for the recognition of optimal $1$-planar graphs.

To Discuss:

\begin{itemize}
  \item Discussion on the desity of simple.
  \item Relation between optimal 2-planar and bar-1-visible.
  \item Relation between optimal 2-planar and 1-bend RAC.
\end{itemize}

% ============================================================================
\bibliographystyle{abbrv}
\bibliography{references}
% ============================================================================

\end{document}
