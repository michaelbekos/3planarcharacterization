% ============================================================================
\section{Characterization of 2-planar graphs}
\label{sec:2planar}
% ============================================================================

Let $G$ be an optimal $2$-planar graph on $n$ vertices (and therefore with $5n-10$ edges). Let also $\Gamma(G)$ be a PMCM $2$-planar drawing of $G$, i.e. $\Gamma(G)$ has the maximum number of true-planar edges among all potential $2$-planar drawing of $G$ and, subject to this restriction, $\Gamma(G)$ has also the minimum number of crossings. 
%For the sake of simplicity we also momentarily assume that in $\Gamma(G)$ there is no pair of edges that cross twice, i.e. $\Gamma(G)$ is almost-simple. This assumption  will be settled soon (see Property~\ref{prp:2_planar_cross_twice}). 
In the following lemmas, we examine structural properties of a PMCM-drawing $\Gamma(G)$.

Note that in a PMCM-drawing $\Gamma(G)$, whenever edge $(w,w')$ has at least $2$ 
crossings, say $c$ and $c'$, then, by 
Property~\ref{prp:2_planar_cross_twice}  there 
exist two (non identical) edges $(u,u')$ and $(v,v')$ that cross $(w,w')$ at $c$ 
and $c'$ respectively. Property~\ref{prp:2_planar_quasi} implies that 
vertices $u$ and $v$ always form a parallel pair, and so do vertices $u'$ and 
$v'$. 
%Since this property will be heavily used throughout the paper, we shall not explicitly state it every time, but rather implicitly imply it.

%\todo[inline]{add in every lemma the assumption about edges crossing twice}

\begin{lemma}
Let $\Gamma(G)$ be a PMCM $2$-planar drawing of an optimal $2$-planar graph $G$. Any edge that is crossed twice in  $\Gamma(G)$ is a chord of a true-planar $5$-cycle in $\Gamma(G)$. 
\label{lem:2_planar_small_faces}
\end{lemma}

\begin{proof}
Let $(w,w')$ be an edge of $G$ that is crossed twice in $\Gamma(G)$ and let 
$(u,u')$ and $(v,v')$ be the corresponding edges crossed by $(w,w')$.\todo{fix 
the notation of edges uniformly} This crossing configuration defines four 
corner pairs: \todo{fix a notation for corner and parallel pairs}$\langle w,u \rangle$, 
$\{w,u'\}$, $\{w,v\}$, $\{w,v'\}$, and two parallel pairs: $\{u,v\}$ and 
$\{u',v'\}$\todo{a ref to a fig is needed for the placement of crossings. Are 
there other x-ing configurations?}. By Properties~\ref{prp:corner} and 
~\ref{prp:parallel}, all corner edges are \pes, and either one or two of the 
parallel edges are \pes. We distinguish two cases depending on whether both 
parallel edges are \pes or not.

So suppose first, that both parallel edges $[u,v]$ and $[u',v']$ are \pes, as 
in Figure~\ref{fig:2_planar_both_parallel_before}. Then vertices 
$w,u,v,w',v',u'$ define a \pp $\mathcal{C}$ of six vertices (grey shaded in 
Figure~\ref{fig:2_planar_both_parallel_before}). There exist at most five edges 
of $\Gamma(G)$ that pass through the interior of $\mathcal{C}$: edges $(w,w')$, 
$(u,u')$, $(v,v')$ and at most two other edges that cross $(u,u')$ or $(v,v')$. 
%Note that since the \pr is an open topological region without vertices in its interior, there can not be an edge that lies entirely in its interior. 
We proceed by removing  edges $(w,w')$, $(u,u')$, $(v,v')$ and all other edges 
that cross $(u,u')$ or $(v,v')$, and replace them with the $2$-planar pattern of 
Figure~\ref{fig:2_planar_both_parallel_after}\todo{use 
Lemma~\ref{lem:exchange}}. In the derived graph there exist $6$ edges drawn 
inside $\mathcal{C}$ and do not cross its boundary. Hence, the derived graph has 
more edges than $G$; a contradiction to the optimality of $G$. \todo{this 
sentence is not needed. It should be obvious so far!}Note that even in the case 
where $\mathcal{C}$ is not simple, i.e. the vertices that define its boundary 
are not all distinct, the above argument still holds, as can be seen for example 
in Figure~\ref{fig:2_planar_both_parallel_non_simple}, where $w=u$ and $v=v'$.
\todo{fix the figure}

%So suppose first, that both parallel edges $(u,v)$ and $(u',v')$ are \pes, as in Figure~\ref{fig:2_planar_both_parallel_before}. Then vertices $w,u,v,w',v',u'$ define a \pp $\mathcal{C}_6$ of six vertices (grey shaded in Figure~\ref{fig:2_planar_both_parallel_before}). The \pr $\mathcal{R}_6$ of $\mathcal{P}_6$ contains no vertices in its interior, and at most five edges of $\Gamma(G)$ pass through $\mathcal{R}_6$: edges $(w,w')$, $(u,u')$, $(v,v')$ and at most two other edges that cross $(u,u')$ or $(v,v')$. Note that since the \pr is an open topological region without vertices in its interior, there can not be an edge that lies entirely in its interior. We proceed by removing  edges $(w,w')$, $(u,u')$, $(v,v')$ and all other edges that cross $(u,u')$ or $(v,v')$, and replace them with the $2$-planar pattern of Figure~\ref{fig:2_planar_both_parallel_after}. In the derived graph there exist $6$ edges drawn in $\mathcal{R}_6$ and do not cross the boundary of the \pp. Hence, the derived graph has more edges than $G$; a contradiction to the optimality of $G$. Note that even in the case where the \pp is not simple, i.e. the vertices that define its boundary are not all distinct, the above argument still holds, as can be seen for example in Figure~\ref{fig:2_planar_both_parallel_non_simple}, where $w=u$ and $v=v'$.
%\todo{fix the figure}

Suppose now that parallel edge $[u,v]$ is not a \pe. Then, it is $u=v$ and 
$[u,v]$ is an homotopic self-loop; refer to 
Figure~\ref{fig:2_planar_one_parallel_before}. This time, vertices 
$w,u,w',v',u'$ define a \pp $\mathcal{C_5}$ of five vertices (grey shaded in 
Figure~\ref{fig:2_planar_one_parallel_before}). As in the previous case, at most 
five edges of $\Gamma(G)$ pass through the interior of $\mathcal{C}_5$. We 
proceed by removing  edges $(w,w')$, $(u,u')$, $(v,v')$ and all other edges that 
cross $(u,u')$ or $(v,v')$, and replace them with the $2$-planar pattern of 
Figure~\ref{fig:2_planar_one_parallel_after}. In the derived graph there exist 
$5$ edges drawn inside $\mathcal{C}_5$ plus the five \pes of its boundary. Since 
$G$ is optimal, it follows that:
\begin{enumerate}
\item there exist exactly two edges other than $(w,w')$, say $e$ and $e'$, that cross $(u,v)$ and $(u',v')$ respectively, and
\item  the \pes of the boundary of  $\mathcal{C}_5$ already exist in the drawing $\Gamma(G)$, i.e. $\mathcal{C}_5$ is a cycle of length $5$. 
\end{enumerate}

%Suppose now that parallel edge $(u,v)$ is not a \pe. Then, it is $u=v$; refer to Figure~\ref{fig:2_planar_one_parallel_before}. This time, vertices $w,u,w',v',u'$ define a \pp $\mathcal{P}_5$ of five vertices (grey shaded in Figure~\ref{fig:2_planar_one_parallel_before}). As in the previous case, at most five edges of $\Gamma(G)$ pass through the polygonal region $\mathcal{R}_5$. We proceed by removing  edges $(w,w')$, $(u,u')$, $(v,v')$ and all other edges that cross $(u,u')$ or $(v,v')$, and replace them with the $2$-planar pattern of Figure~\ref{fig:2_planar_one_parallel_after}. In the derived graph there exist $5$ edges drawn in $\mathcal{R}_5$ plus the five \pes of the boundary of $\mathcal{P}_5$. Since $G$ is optimal, it follows that:
%\begin{enumerate}
%\item there exist exactly two edges other than $(w,w')$, say $e_1$ and $e_2$, that cross $(u,v)$ and $(u',v')$ respectively, and
%\item  the \pes of the boundary of  $\mathcal{P}_5$ already exist in the drawing $\Gamma(G)$, i.e. $\mathcal{P}_5$ is a cycle, say $C$, of length $5$. 
%\end{enumerate}

If $\mathcal{C}_5$ is a true-planar $5$-cycle in $\Gamma(G)$ the lemma holds, so 
suppose that this is not the case. Then, at least one of edges $e$ or $e'$ 
crosses a boundary edge of $\mathcal{C}_5$. Suppose w.l.o.g. that  edge 
$e=(x,x')$ crosses edge $(v,v')$ of $\mathcal{C}_5$ at crossing point $c$ (the 
case where $e$ crosses edge $(u,v)$ can be treated similarly);\todo{is it clear 
that it can't cross another edge of C?} refer to 
Figure~\ref{fig:2_planar_one_parallel_extra}. Note that $e$ already has two 
crossings in the drawing $\Gamma(G)$. Then, at least one of the edge segments 
$(x,c)$ or $(c,x')$ of $e$ does not pass through the interior of 
$\mathcal{C}_5$. Suppose w.l.o.g. that this is edge segment $(x,c)$  of $e$. 
Then, vertices \todo{use uniform notation}$x$ - $v$, and $x$ - $v'$ define two 
corner pairs of vertices. Hence vertices $w,u,w',v',x,v$ define a \pp 
$\mathcal{C}_6$ on six vertices, with exactly six edges passing through its 
interior. We remove all edges that pass through $\mathcal{C}_6$ and replace it 
with the $2$-planar pattern of Figure~\ref{fig:2_planar_one_parallel_final}: we 
add one vertex and a total of $12$ edges\todo{use Lemma~\ref{lem:exchange}}. 
The derived graph, say $G'$, is $2$-planar and has $n'=n+1$ vertices and 
$m'=m+6$ edges (where $n$, $m$ are the number of vertices and edges of $G$ 
respectively). Hence $m'=5n'-9>5n'-10$, i.e. $G'$ has more edges than allowed; a 
contradiction.


%If $C$ is a true-planar $5$-cycle in $\Gamma(G)$ the lemma holds, so suppose that this is not the case. Then, at least one of edges $e_1$ or $e_2$ crosses $C$. Suppose w.l.o.g. that  edge $e_1=(w_1,w'_1)$ crosses edge $(v,v')$ of the $5$-cycle $C$ at crossing point $c$ (the case where $e_1$ crosses edge $(u,v)$ can be treated similarly);\todo{is it clear that it can't cross another edge of C?} refer to Figure~\ref{fig:2_planar_one_parallel_extra}. Note that $e_1$ already has two crossings in the drawing $\Gamma(G)$. Then, at least one of the edge segments $(w_1,c)$ or $(c,w'_1)$ of $e_1$ does not pass through the \pr $\mathcal{R}_5$. Suppose w.l.o.g. that this is edge segment $(w_1,c)$  of $e_1$. Then, vertices $w_1$ - $v$, and $w_1$ - $v'$ define two corner pairs of vertices. Hence vertices $w,u,w',v',w_1,v$ define a \pp $\mathcal{P}_6$ on six vertices, with exactly six edges passing through its \pr, say $\mathcal{R}_6$. We remove all edges that pass through $\mathcal{R}_6$ and replace it with the $2$-planar pattern of Figure~\ref{fig:2_planar_one_parallel_final}: we add one vertex and a total of $12$ edges. The derived graph, say $G'$, is $2$-planar and has $n'=n+1$ vertices and $m'=m+6$ edges (where $n$, $m$ are the number of vertices and edges of $G$ respectively). Hence $m'=5n'-9>5n'-10$, i.e. $G'$ has more edges than allowed; a contradiction.


\end{proof}

\begin{figure}[htb]
    \centering
    \begin{minipage}[b]{.24\textwidth}
        \centering
        \includegraphics[width=\textwidth,page=1]{images/2_planar_potential_parallel}
        \subcaption{~}\label{fig:2_planar_both_parallel_before}
    \end{minipage}
    \begin{minipage}[b]{.24\textwidth}
        \centering
        \includegraphics[width=\textwidth,page=2]{images/2_planar_potential_parallel}
        \subcaption{~}\label{fig:2_planar_both_parallel_after}
    \end{minipage}
	\begin{minipage}[b]{.24\textwidth}
        \centering        
        \includegraphics[width=\textwidth,page=3]{images/2_planar_potential_parallel}
        \subcaption{~}\label{fig:2_planar_both_parallel_non_simple}
    \end{minipage}
		
    \begin{minipage}[b]{.24\textwidth}
        \centering
        \includegraphics[width=\textwidth,page=4]{images/2_planar_potential_parallel}
        \subcaption{~}\label{fig:2_planar_one_parallel_before}
    \end{minipage}
		\begin{minipage}[b]{.24\textwidth}
        \centering
        \includegraphics[width=\textwidth,page=5]{images/2_planar_potential_parallel}
        \subcaption{~}\label{fig:2_planar_one_parallel_after}
    \end{minipage}
		\begin{minipage}[b]{.24\textwidth}
        \centering
        \includegraphics[width=\textwidth,page=6]{images/2_planar_potential_parallel}
        \subcaption{~}\label{fig:2_planar_one_parallel_extra}
    \end{minipage}
		\begin{minipage}[b]{.24\textwidth}
        \centering
        \includegraphics[width=\textwidth,page=7]{images/2_planar_potential_parallel}
        \subcaption{~}\label{fig:2_planar_one_parallel_final}
    \end{minipage}
    \caption{%
    Different configurations used in Lemma~\ref{lem:2_planar_small_faces}.}
    \label{fig:2_planar_potential_parallel}
\end{figure}




By Lemma~\ref{lem:2_planar_small_faces}, we have that any edge of $G$ that is 
crossed twice in a PMCM-drawing $\Gamma(G)$ is a chord of a true-planar 
$5$-cycle. So, it remains to consider edges of $G$ that have only one crossing 
in $\Gamma(G)$. In fact, the following lemma states that there there are 
no such edges in the drawing.

\begin{lemma}\label{lem:2_planar_one_crossing}
Let $\Gamma(G)$ be a PMCM $2$-planar drawing of an optimal $2$-planar graph $G$. Then every edge of $\Gamma(G)$ is either true-planar or has exactly two crossings.
\end{lemma}
\begin{proof}
By the proof of Lemma~\ref{lem:2_planar_small_faces}, we have that, for any edge 
of $G$ that is crossed twice in $\Gamma(G)$, both edges that cross this 
particular edge also have two crossing in $\Gamma(G)$. In other words, the 
crossing component of an edge with two crossings, contains only edges with two 
crossings. This implies that if edges $(u,u')$ and $(v,v')$ cross in $\Gamma(G)$ 
and $(u,u')$ has only one crossing, then the same holds for $(v,v')$. \todo{use 
crossing component}So, we have four corner pairs of  vertices: \todo{use 
uniform notation}$u$ - $v$, $u$ - $v'$, $u'$ - $v$, $u'$ - $v'$, and all corner 
edges are \pes; refer to Figure~\ref{fig:2_planar_one_crossing_before}. Vertices 
 $u,v,v',u'$ define a \pp $\mathcal{C}$ of four vertices (grey shaded in 
Figure~\ref{fig:2_planar_one_crossing_before}). Since edges $(u,u')$ and 
$(v,v')$ have only one crossing each, the \pes of the boundary of $\mathcal{C}$ 
exist in $\Gamma(G)$ and are true-planar edges (Indeed if any of these \pes does 
not exist, one could add them in $\Gamma(G)$ without introducing any crossings 
or creating homotopic edges, contradicting the optimality of $G$). We proceed by 
removing  edges $(u,u')$ and $(v,v')$, and replace them with the $2$-planar 
pattern of Figure~\ref{fig:2_planar_one_crossing_after}. The derived graph, say 
$G'$ is $2$-planar and has $n'=n+2$ vertices and $m'=m+11$ edges (where $n$, $m$ 
are the number of vertices and edges of $G$ respectively). Hence 
$m'=5n'-9>5n'-10$, i.e. $G'$ has more edges than allowed; a contradiction.
%By the proof of Lemma~\ref{lem:2_planar_small_faces}, we have that, for any edge of $G$ that is crossed twice in $\Gamma(G)$, both edges that cross this particular edge also have two crossing in $\Gamma(G)$. This implies that if edges $(u,u')$ and $(v,v')$ cross in $\Gamma(G)$ and $(u,u')$ has only one crossing, then the same holds for $(v,v')$. \todo{use propagated set}So, we have four corner pairs of  vertices: $u$ - $v$, $u$ - $v'$, $u'$ - $v$, $u'$ - $v'$, and all corner edges are \pes; refer to Figure~\ref{fig:2_planar_one_crossing_before}. Vertices  $u,v,v',u'$ define a \pp $\mathcal{P}_4$ of four vertices (grey shaded in Figure~\ref{fig:2_planar_one_crossing_before}). Since edges $(u,u')$ and $(v,v')$ have only one crossing each, the \pes of the boundary of $\mathcal{P}_4$ exist in $\Gamma(G)$ and are true-planar edges (Indeed if any of these \pes does not exist, one could add them in $\Gamma(G)$ without introducing any crossings or creating homotopic edges, contradicting the optimality of $G$). We proceed by removing  edges $(u,u')$ and $(v,v')$, and replace them with the $2$-planar pattern of Figure~\ref{fig:2_planar_one_crossing_after}. The derived graph, say $G'$ is $2$-planar and has $n'=n+2$ vertices and $m'=m+11$ edges (where $n$, $m$ are the number of vertices and edges of $G$ respectively). Hence $m'=5n'-9>5n'-10$, i.e. $G'$ has more edges than allowed; a contradiction.

%By the proof of Lemma~\ref{lem:2_planar_small_faces}, we have that, for any edge of $G$ that is crossed twice in $\Gamma(G)$, both edges that cross this particular edge also have two crossing in $\Gamma(G)$. This implies that if edges $(u_1,v_1)$ and $(u_2,v_2)$ cross in $\Gamma(G)$ and $(u_1,v_1)$ has only one crossing, then the same holds for $(u_2,v_2)$. So, we have four corner pairs of  vertices: $u_1$ - $u_2$, $u_1$ - $v_2$, $v_1$ - $u_2$, $v_1$ - $v_2$, and all the potential corner edges exist; refer to Figure~\ref{fig:2_planar_one_crossing_before}. Vertices  $u_1,u_2,v_2,v_1$ define a \pp $\mathcal{P}_4$ of four vertices (grey shaded in Figure~\ref{fig:2_planar_one_crossing_before}). Since edges $(u_1,v_1)$ and $(u_2,v_2)$ have only one crossing each, the \pes of the boundary of $\mathcal{P}_4$ exist in $\Gamma(G)$ and are true-planar edges (Indeed if any of these \pes does not exist, one could add them in $\Gamma(G)$ without introducing any crossings or creating homotopic edges, contradicting the optimality of $G$). We proceed by removing  edges $(u_1,v_1)$ and $(u_2,v_2)$, and replace them with the $2$-planar pattern of Figure~\ref{fig:2_planar_one_crossing_after}. The derived graph, say $G'$ is $2$-planar and has $n'=n+2$ vertices and $m'=m+11$ edges (where $n$, $m$ are the number of vertices and edges of $G$ respectively). Hence $m'=5n'-9>5n'-10$, i.e. $G'$ has more edges than allowed; a contradiction.\qed
\end{proof}


\begin{figure}[htb]
    \centering
    \begin{minipage}[b]{.24\textwidth}
        \centering
        \includegraphics[width=\textwidth,page=1]{images/2planar_one_crossing}
        \subcaption{~}\label{fig:2_planar_one_crossing_before}
    \end{minipage}
    \begin{minipage}[b]{.24\textwidth}
        \centering
        \includegraphics[width=\textwidth,page=2]{images/2planar_one_crossing}
        \subcaption{~}\label{fig:2_planar_one_crossing_after}
    \end{minipage}
    \caption{%
    Configurations used in Lemma~\ref{lem:2_planar_one_crossing}}.
    \label{fig:2_planar_one_crossing}
\end{figure}


 
Now we are ready to prove the main property of every  PMCM-drawing $\Gamma(G)$ of a maximal $2$-planar graph. By Lemmas~\ref{lem:2_planar_small_faces} and \ref{lem:2_planar_one_crossing} we have that there exist only edges with two crossings and true-planar edges in $\Gamma(G)$ that create only true-planar $5$-cycles. 
We claim that the true planar skeleton $\Pi(G)$ is connected. Suppose that this is not the case, and that $\Pi(G)$ is not connected. Since $G$ is connected, there exist two connected components $\Pi(G_1)$ and $\Pi(G_2)$ of $\Pi(G)$ and an edge $(u_1,u_2)$ in $G$ such that $u_1\in V(G_1)$ and $u_2\in V(G_2)$. Clearly  $(u_1,u_2)$ is not true-planar, but by Lemma~\ref{lem:2_planar_small_faces} there exists a true-planar $5$-cycle with $(u_1,u_2)$ drawn as a chord. This implies that there exists a path of true-planar edges of $\Pi(G)$ connecting vertices $u_1$ and $u_2$; a contradiction. Therefore, we have proven the following:
 

 \begin{corollary}\label{cor:2_planar_faces}
  The true planar skeleton $\Pi(G)$ of a PMCM-drawing $\Gamma(G)$ of a maximal $2$-planar graph $G$ contains only faces of length $5$.
 \end{corollary}



%Recall that at the beginning of this section we made the assumption that there is no pair of edges that cross twice in the drawing $\Gamma(G)$. In the following, we prove that in a planar-maximal crossing-minimal drawing there are no such pairs of edges.



%\begin{lemma}
%Let $\Gamma(G)$ be a $2$-planar drawing of an optimal $2$-planar graph $G$ on $n$ vertices. There is no pair of edges that cross twice in $\Gamma(G)$. 
%\label{lem:2_planar_cross_twice}
%\end{lemma}
%
%\begin{proof}
%Suppose that there exist edges $(u_1,v_1)$ and $(u_2,v_2)$ that cross twice at crossing points $c_1$ and $c_2$. Consider the bounded region $R$ defined by $(c_1,c_2)$ segment of edges $(u_1,v_1)$ and $(u_2,v_2)$. Since edges $(u_1,v_1)$ and $(u_2,v_2)$ both have two crossings, no other edge of $G$ crosses the boundary of $R$. Hence, if there exists at least one vertex of $G$ inside this region, $G$ is not connected; a contradiction.\todo{mention that G is connected} On the other hand, if $R$ is empty, we could redraw edges $(u_1,v_1)$ and $(u_2,v_2)$ so that they do not cross; a contradiction to the fact that $\Gamma(G)$ is crossing-minimal.\qed
%\end{proof}

%By combining Corollary~\ref{cor:2_planar_faces} and Property~\ref{prp:2_planar_cross_twice}, we can characterize all optimal $2$-planar graphs:


 %By Property~\ref{prp:2_planar_cross_twice} we have that Corollary~\ref{cor:2_planar_faces} holds for all maximal $2$-planar graphs. 
Since the true planar skeleton of such a graph contains only faces of length $5$, we can start with a $5$-tiling of the plane\todo{rephrase}. Now in the interior of every face of length $5$, we can add all  $5$ missing edges using the $2$-planar pattern of Figure~\ref{fig:2_planar_one_parallel_after}.

