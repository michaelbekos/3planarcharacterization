% ============================================================================
\section{Introduction}
\label{sec:introduction}
% ============================================================================

The gradually improvement of the well-known lemma of crossing numbers \cite{ACNS82,Lei83}, but also recent interest in graphs beyond planarity gave rise to multiple extensions of planar graphs allowing edge crossing in some restricted local configurations. The most prominent is the concept of $k$-planarity; here each edge can be crossed at mot $k$ times. Earlier work mainly considered the case $k=1$, i.e., $1$-planar graphs. We mention here the coloring problem of 1-planar graphs by Ringel~\cite{Ringel65} and the first density bounds by Bodendiek, Schumacher and Wagner~\cite{BSW84}. Later on, Suzuki \cite{DBLP:journals/siamdm/Suzuki10} gave simple rules how to generate maximal $1$-planar graphs. Other works considered maximal 1-planar graphs~\cite {DBLP:conf/gd/BrandenburgEGGHR12} and recognition of $1$-planar graphs~\cite{DBLP:journals/corr/Brandenburg16a}. But also subclasses have been considered like \emph{IC-planar graphs}, where for each vertex, only one incident edge can be crossed \cite{DBLP:conf/gd/BrandenburgDEKL15}, and several others on outer-1-planarity where all the vertices have to be adjacent to the outer face as being required in the corresponding model of outer-planarity~\cite{DBLP:journals/algorithmica/HongEKLSS15,DBLP:journals/jgaa/GiacomoLM15,DBLP:journals/algorithmica/AuerBBGHNR16}.

For larger $k \geq 2$, the paper of Pach and Toth~\cite{PachT97} provided significant progress, as it gives techniques for upper bounds on the edge density for simple $k$-planar graphs, which then lead to upper bounds of $5n -10$ for $2$-planar, $6n -12$ for $3$-planar and $7n-13$ for $4$-planar graphs. For larger $k$, the authors provide a bound of $4.1 \sqrt k n$. While the bound for $2$-planar graphs is tight, the other bounds have been improved to $5.5n - 11$ \cite{PachRTT06} and $6n-12$ \cite{DBLP:journals/corr/Ackerman15} for simple $3$- and $4$-planar graphs, respectively. In our companion paper~\cite{BKR16}, we generalize the result and the bound of Pach et al.~\cite{PachRTT06} to non-simple graphs, where non-homotopic parallel edges as well as non-homotopic selfloops are allowed.

In this paper, we now completely characterize optimal non-simple $2$-planar and $3$-planar graphs, i.e., those that achieve the bound of $5n-10$ and $5.5n-11$ on the number of edges, respectively. In particular, we prove that the commonly known $2$-planar graphs that achieve the upper bound of $5n-10$ are the only optimal $2$-planar graphs. Such graphs consist of a crossing-free subgraph where all faces have size $5$, which then are enhanced by $5$ more edges forming a $K_5$'s together with the boundary edges of the face (Section~\ref{sec:2planar}). For $3$-planar graphs, we can show correspondingly that optimal $3$-planar graphs have a similar simple structure (Section~\ref{sec:3planar}).

Related to this work are also other extensions of planarity, like fan-planarity \cite{KU14,DBLP:conf/gd/BekosCGHK14,DBLP:journals/tcs/BinucciGDMPST15,DBLP:conf/gd/BinucciCDGKKMT15}, quasi-planarity \cite{AAPPS97,DBLP:journals/cj/GiacomoDLMW15}, k-quasi-planarity~\cite{DBLP:conf/gd/Suk11,DBLP:journals/comgeo/SukW15}, RAC drawability~\cite{DBLP:journals/tcs/DidimoEL11,DBLP:journals/jgaa/ArgyriouBS12}, and fan-crossing-free planarity~\cite{DBLP:journals/algorithmica/CheongHKK15} 
as well as alternative geometric representations like bar-1 visibility~\cite{DBLP:conf/icaa/SultanaRRT14,DBLP:journals/jgaa/Evans0LMW14,DBLP:conf/walcom/BrandenburgEN16} and more in general bar-$k$ visibility~\cite{DBLP:journals/dm/GenesonKT14,DBLP:journals/corr/SawhneyW16}. Note that all aforemention classes of nearly-planar graphs have also linear number of edges (except for the classes of $k$-quasi planar and bar-$k$ visibile graphs, whose number of edges also depends on the parameter $k$). 

